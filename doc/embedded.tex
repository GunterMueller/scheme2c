\documentclass[12pt]{article}
\usepackage{fullpage}
\usepackage{parskip}
\usepackage{newcent}
\usepackage{s2c}

\title{Embedded \StoC\ --- 1 February 1993}
\date{}
\author{Joel F. Bartlett}

\begin{document}

\maketitle

One of the major goals of \StoC\ was to build a Scheme compiler and
runtime system that could coexist with other programming languages.
While this effort has been quite successful, \StoC\ has not offered
everything needed by applications that wish to embed a Scheme server,
feed it arbitrary expressions for evaluation, yet remain responsive to
additional requests.

For example, a database server might want to to evaluate arbitrary
Scheme expressions to verify record update operations, record access
rights, or provide data that is computed rather than being resident in
the database.  Or, an event driven application for a PC or a Macintosh
might like to embed Scheme.

In order to safely execute in such a variety of environments, the
Scheme system must allow the application to handle external events on
a timely basis and place minimum requirements on the available system
services. In order to solve these problems, the 01feb93 release of
\StoC\ has been modified in the following areas.

\begin{description}
\item[Explicit Time Slicing:] In order to return to the caller at
  regular intervals, \StoC\ can be compiled with explicit time
  slicing.  On each procedure entry or backwards branch, a counter is
  decremented. When the counter goes to 0, Scheme returns control to
  the application program at the point where it was called.  At a
  later time, the application has the option of continuing the
  previous computation or starting a new computation.

\item[Explicit Stack Overflow Checks:] \StoC\ may be compiled with
  explicit stack overflow checks.  This is necessary as many
  environments have no other means for detecting a stack overflow
  which could damage the embedding application or crash the personal
  computer.

\item[Request-response interaction:] When used as an embedded server,
  all interaction with Scheme is via one interface procedure.  Errors
  and breakpoints are handled across this interface like any other
  type of request.

\item[Operating System independence:] Unlike previous releases,
  \StoC\ does not assume the existence of a UNIX-like I/O system.
  Rather than directly calling the host I/O system, all requests are
  via implementation specific routines in scrt/cio.c.  In fact, an
  embedded \StoC\ server doesn't assume that the client even has an
  I/O system.  Instead the stdout and stderr ports are string output
  ports.

\item[Traps:] Previous releases of \StoC\ used operating system traps
  to detect division by zero and keyboard interrupts.  Division by
  zero errors are now explicitly tested for, and keyboard interrupt
  signals are not used by embedded \StoC\ systems.
\end{description}

\subsection*{Application interface}

Applications evaluate a Scheme expression by calling the procedure
\texttt{scheme2c}:
\begin{quote}
\begin{verbatim}
void  scheme2c( char *expression,
                int *status,
                char **result,
                char **error )
\end{verbatim}
\end{quote}
where \texttt{expression} is a pointer to a null terminated string of
ASCII characters that is the Scheme expression to evaluate.  When the
procedure returns, the result is stored in \texttt{status},
\texttt{result}, and \texttt{error}.  The value returned in
\texttt{status} is interpreted as follows:

\begin{itemize}
\item[\texttt{0}:] expression evaluated normally.  The value is saved
  in \texttt{*SCHEME2C-RESULT*} within the Scheme system and it is
  also written to Scheme's stdout-port.

\item[\texttt{1}:] an error occurred.  The error message is written to
  Scheme's stderr-port.  If no previous error is being examined, the
  stack trace is written to Scheme's stderr-port and the associated
  environments are in the list \texttt{*ERROR-ENV*}.  The client
  should evaluate \texttt{(RESET-ERROR)} when done examining the error
  state.  Note that if additional errors occur before
  \texttt{(RESET-ERROR)} is evaluated, they will not cause a stack
  dump, nor have the error environment saved.

\item[\texttt{2}:] an internal error in \StoC\ occurred.  The error
  message is reported via Scheme's stderr-port.  No further execution
  is possible.

\item[\texttt{3}:] the computation timed out.  Evaluate
  \texttt{(PROCEED)} to continue execution.  Evaluate
  \texttt{(PROCEED?)} to cause a breakpoint when execution resumes.

\item[\texttt{4}:] a procedure entry breakpoint occurred.  The call
  arguments are written to Scheme's stderr-port and the associated
  environments are in the list \texttt{*BPT-ENV*}. The procedure stack
  trace can be viewed by evaluating \texttt{(BACKTRACE)}.  The
  procedure arguments are in \texttt{*ARGS*}. Evaluate
  \texttt{(PROCEED)} to continue execution, or \texttt{(RESET-BPT)} to
  abort.

\item[\texttt{5}:] a procedure exit breakpoint occurred.  The result
  is written to Scheme's stderr-port and saved in \texttt{*RESULT*}.
  The environments are in the list \texttt{*BPT-ENV*}.  Evaluate
  \texttt{(PROCEED)} to continue execution, \texttt{(PROCEED
    \textnormal{\emph{expression}})} to continue returning a new
  value, or \texttt{(RESET-BPT)} to abort.  Note that additional
  breakpoints will not occur while examining the state of a
  breakpoint.
\end{itemize}

The value returned in \texttt{result} is pointer to a null terminated
string of ASCII characters that is the contents of Scheme's
stdout-port, i.e. the standard output port.  The value retirned in
\texttt{error} is a pointer to a null terminated string of ASCII
characters that is the contents of Scheme's stderr-port, i.e. the
error output port.

The stack size which is used by an embedded Scheme system is set by
evaluating \texttt{(set-stack-size!\ \textnormal{\emph{expression}})},
where \emph{expression} is the size in bytes.  The current stack size
can be obtained by evaluating \texttt{(stack-size)}.  Scheme reserves
a portion of this stack for error recovery, but this may be exceeded
on some implementations.  An application should verify that it has set
the stack correctly by evaluating an expression that forces a stack
overflow error and then verifying that the Scheme stack has not
overflowed into the application.

The Scheme time slice is set by evaluating
\texttt{(set-time-slice!\ \textnormal{\emph{expression}})}, where
\emph{expression} is the number of Scheme procedure calls that should
be made in a time slice.  Experiment to find the right value for your
application. The current time slice value is obtained by evaluating
\texttt{(time-slice)}.

\subsection*{Sample embedded application}

A sample embedded Scheme application, \texttt{embedded}, is found in
the directory \texttt{server}.  A typescript of its execution shows
how an application should interact with an embedded Scheme system.
\begin{quote}
\begin{verbatim}
csh 990 >embedded
Embedded Scheme->C Test Bed
0- (+ 1 2)
3
0-
\end{verbatim}
\end{quote}
The program prompts the user for a Scheme expression which is then
evaluated using \texttt{scheme2c}.  The \texttt{result} and
\texttt{error} messages are then printed, followed by a prompt
(incorporating the value of \texttt{status}) for the next expression.
\begin{quote}
\begin{verbatim}
0- (time-slice)
100000
0- (stack-size)
57000
0- (let loop ((i 0)) (loop (+ i 1)))
3- (proceed)
3- (proceed)
3- (proceed?)
(+ I 1) in ENV-0
(LOOP (+ I 1)) in ENV-1
(EVAL ...)
(EXECUTE [inside SCHEME2C] ...)
(SCREP_SCHEME2C ...)
4-
\end{verbatim}
\end{quote}
After obtaining the current time slice and stack size values, a
looping expression was entered.  Twice it timed out and was continued
by evaluating the expression \texttt{(proceed)}.  The third time it
timed out, \texttt{(proceed?)} was entered to force a breakpoint.
\begin{quote}
\begin{verbatim}
4- (list-ref *bpt-env* 0)
((I . 21345) (LOOP . #*PROCEDURE*) ($_0 . 0))
0- (proceed)
3- (proceed?)
(LOOP (+ I 1)) in ENV-0
(EVAL ...)
(EXECUTE [inside SCHEME2C] ...)
(SCREP_SCHEME2C ...)
4-  (list-ref *bpt-env* 0)
((I . 28476) (LOOP . #*PROCEDURE*) ($_0 . 0))
0- (reset-bpt)
#F
0-
\end{verbatim}
\end{quote}
The environment at the time of the breakpoint was examined to find the
value of \texttt{i} by looking at element 0, corresponding to
\texttt{env-0}, of the list \texttt{*bpt-env*}.  The program was then
contined by \texttt{(proceed)} and then \texttt{i} was examined at the
end of the next time slice to find out how much work was done.
\begin{quote}
\begin{verbatim}
0- (let ((x 1) (y 2)) (+ x y z))
***** Z Top-level symbol is undefined
(+ X Y Z) in ENV-0
(EVAL ...)
(EXECUTE [inside SCHEME2C] ...)
(SCREP_SCHEME2C ...)
1- *bptenv*
***** *BPTENV* Top-level symbol is undefined
0- *error-env*
(((Y . 2) (X . 1)))
0- (reset-error)
#F
0-
\end{verbatim}
\end{quote}
An error occurred and then error environment was examined.  While
examining the error environment, another error occurred.  This simply
resulted in an error message.
\begin{quote}
\begin{verbatim}
0- (define (f x) (* 2 x))
F
0- (bpt f)
F
0- (f 23)
0 -calls  - (F 23)
4- (proceed)
0 -returns- 46
5- (proceed)
46
0- (f 40)
0 -calls  - (F 40)
4- (f 10)
20
0- (proceed)
0 -returns- 80
5- (proceed 15)
15
0-
\end{verbatim}
\end{quote}
The final example shows the use of procedure breakpoints.  One thing
to note, is that breakpoints do not nest.

\subsection*{Adding Your Code to an Embedded \StoC\ System}

There are two ways to do this.  The easiest is to replace the module
scrtuser (in the files scrtuser.sc and scrtuser.c) and then rebuild
the embedded Scheme system.  An alternative is to separately compile
your modules, link them into your application, and then explicitly
initialize your modules after Scheme has been initialized.  This is
done by calling \texttt{scheme2c} with an expression like
\texttt{"\#t"} and then calling your module initialization procedures
which have the form \emph{module-name}\texttt{\_\_init}.
\end{document}
