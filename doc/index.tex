\documentclass[10pt,twocolumn]{article}
\usepackage{fullpage}
\usepackage{parskip}
\usepackage{textcomp}
\usepackage{newcent}
\usepackage{s2c}

\title{\StoC\ Index to the\\
Revised$^4$ Report on the Algorithmic Language Scheme}
\author{Joel F. Bartlett}
\date{15 March 1993}

\begin{document}

\maketitle

\section*{Implementation Notes}

\StoC\ is an implementation of the language Scheme as
described in the \emph{Revised$^4$ Report on the Algorithmic Language
Scheme} (\emph{LISP Pointers}, Volume IV, Number 3, July-September 1991).

The implementation is known to not conform to the required portions of
the report in the following ways:

\begin{itemize}
\item
The syntax for numbers reflects the underlying C implementation.
Scheme programs may not use the numeric prefixes \texttt{\#i}
and \texttt{\#e}, and numbers may not contain \texttt{\#} as a digit.

\item
Numerical input and output uses the facilities of the
underlying C implementation.  As a result, the constraints of
section 6.5.6 may not be satisfied.

\item
As /, quotient, and remainder depend upon C's behavior for
negative fixed arguments (which is undefined), those doing ports
must verify their correct operation.

\item
Implementations that do not handle arithmetic overflow traps
may return incorrect results when an overflow occurred during
the operation.

\item
The control flow of compiled programs is constrained by the
underlying C implementation.  As a result, some tail calls are not compiled as
tail calls.
\end{itemize}

The implementation has been extended beyond the report in the
following ways:

\begin{itemize}

\item Additional procedures:
\begin{verbatim}
%list->record
%record
%record->list
%record-length
%record-lookup-method
%record-methods
%record-methods-set!
%record-ref    %record-set!
%record?
after-collect
backtrace
bit-and        bit-lsh
bit-not        bit-or
bit-rsh        bit-xor
c-byte-ref     c-byte-set!
c-double-ref   c-double-set!
c-float-ref    c-float-set!
c-int-ref      c-int-set!
c-longint-ref  c-longint-set!
c-longunsigned-ref
c-longunsigned-set!
c-s2cuint-ref  c-s2cuint-set!
c-shortint-ref  
c-shortint-set!
c-shortunsigned-ref
c-shortunsigned-set!
c-sizeof-double
c-sizeof-float
c-sizeof-int   c-sizeof-long
c-sizeof-s2cuint
c-sizeof-short c-sizeof-tscp
c-string->string
c-tscp-ref     c-tscp-set!
c-unsigned-ref
c-unsigned-set!
catch-error    close-port
collect        collect-all        
collect-info   cons*
define-system-file-task
echo
enable-sytem-file-tasks
error          eval
exit
expand         expand-once
fixed->float   fixed?
float->fixed   float?
flush-buffer   format
get-output-string
getprop        getprop-all
implementation-information
last-pair
open-file
open-input-string
open-output-string
optimize-eval
port->stdio-file
pp
proceed        proceed?
putprop
read-eval-print
remove         remove!
remq           remq!
remv           remv!
remove-file    rename-file
reset
reset-bpt      reset-error
scheme-byte-ref
scheme-byte-set!
scheme-int-ref
scheme-int-set!
scheme-s2cuint-ref
scheme-s2cuint-set!
scheme-tscp-ref
scheme-tscp-set!
set-gcinfo!
set-generation-limit!
set-maximum-heap!
set-stack-size!
set-time-slice!
set-top-level-value!
set-write-circle!
set-write-length!
set-write-level!
set-write-pretty!
set-write-width!
signal
stack-size
string->uninterned-symbol
system
time-of-day
time-slice
top-level      
top-level-value
uninterned-symbol?
wait-system-file
weak-cons
when-unreferenced
write-circle   write-count
write-length   write-level
write-pretty   write-width
\end{verbatim}

\item Additional syntax:
\begin{verbatim}
bpt
define-c-external
define-constant
define-external
define-in-line
define-macro   include
module
trace          untrace
unbpt
unless         when
\end{verbatim}

\item Additional variables:
\begin{verbatim}
%record-prefix-char
%record-read
*args*         *bpt-env*
debug-output-port
*error-env*
*error-handler*
*frozen-objects*
*obarray*      *result*
*scheme2c-result*
stderr-port    stdin-port
stdout-port
trace-output-port
\end{verbatim}
\end{itemize}

\section*{Index}

\texttt{"} delimits strings.  Inside a string constant, a \texttt{"}
is represented by \texttt{\textbackslash"}, and a
\texttt{\textbackslash} is represented by
\texttt{\textbackslash\textbackslash}.  \RRRRRS~25.

\texttt{\#(} denotes the start of a vector.  \RRRRRS~26.

\texttt{\#\textbackslash}\emph{character} written notation for
characters.  \RRRRRS~24.

\texttt{\#\textbackslash{}formfeed} ASCII form feed character (\#o14).
\RRRRRS~24.

\texttt{\#\textbackslash{}linefeed} ASCII line feed character (\#o12).
\RRRRRS~24.

\texttt{\#\textbackslash{}newline} new line character (\#o12).
\RRRRRS~24.

\texttt{\#\textbackslash{}return} ASCII carriage return character
(\#o15).  \RRRRRS~24.

\texttt{\#\textbackslash{}space} ASCII space character (\#o40).
\RRRRRS~24.

\texttt{\#\textbackslash{}tab} ASCII tab character (\#o11).
\RRRRRS~24.

\texttt{\#b} binary radix prefix.  \RRRRRS~20.

\texttt{\#d} decimal radix prefix.  \RRRRRS~20.

\texttt{\#f} boolean constant for false.  Note that while the empty
list \texttt{()} is also treated as a false value in conditional
expressions, it is not the same as \texttt{\#f}.  \RRRRRS~13.

\texttt{\#o} octal radix prefix.  \RRRRRS~20.

\texttt{\#t} boolean constant for true.  \RRRRRS~13.

\texttt{\#x} hex radix prefix.  \RRRRRS~20.

(\texttt{\%list->record} \emph{list}) returns a newly created
\emph{record} whose elements are the members of \emph{list}.

(\texttt{\%record} \emph{expression} ...)\ returns a newly created
\emph{record} whose elements contain the given arguments.

(\texttt{\%record->list} \emph{record}) returns a newly created
\emph{list} of the objects contained in the elements of \emph{record}.

(\texttt{\%record-length} \emph{record}) returns the number of
elements in \emph{record}.

(\texttt{\%record-lookup-method} \emph{record} \emph{method}) returns
either the \emph{record}'s method \emph{procedure} or \texttt{\#f}
when no method is defined for the method named \emph{method}.  All
records have defaults for the following methods:
\texttt{\%to-display}, \texttt{to-equal?}, \texttt{\%to-eval}, and
\texttt{\%to-write}.

(\texttt{\%record-methods} \emph{record}) returns a list of
\emph{pairs} that denote the methods for \emph{record}.  Each
\emph{pair} is composed of a \emph{symbol} denoting the method name
and the method \emph{procedure}.

(\texttt{\%record-methods-set!}\ \emph{record} \emph{methods}) sets
the methods associated with \emph{record} to \emph{methods}, a
\emph{list} of method \emph{pairs}.

\texttt{\%record-prefix-char} is the character that denotes a
\emph{record}.

\texttt{\%record-read} is a \emph{procedure} that is called to read a
\emph{record}. When \texttt{read} encounters the value of
\texttt{\%read-prefix-char} following a \texttt{\#}, it calls
\texttt{\%record-read} with the current \emph{input-port} as its
argument to input the record.  The value read is the value returned by
this procedure.

(\texttt{\%record-ref} \emph{record} \emph{integer}) returns the
contents of element \emph{integer} of \emph{record}.  The first
element is 0.

(\texttt{\%record-set!}\ \emph{record} \emph{integer}) sets element
\emph{integer} of \emph{record} to \emph{expression}.

(\texttt{\%record?}\ \emph{expression}) \emph{predicate} that returns
\texttt{\#t} when \emph{expression} is a \emph{record}.

\texttt{\%to-display} method to \texttt{display} a \emph{record}.
When \texttt{display} encounters a record, it calls the record's
\texttt{\%to-display} method with the following arguments: the record,
the output port, the number of spaces to indent (or \texttt{\#f}), the
number of levels to print (or \texttt{\#f}), the length of lists,
vectors, or records to print (or \texttt{\#f}), and a list of pairs,
vectors, and records already seen (or \texttt{\#f}).  The method
returns either \texttt{\#f} indicating no further action is to be
taken, or a pair indicating that the car of the pair is to be output.
For example, if \texttt{\%record-prefix-char} is
\texttt{\#\textbackslash\texttildelow}, the method could be:
\texttt{(lambda (r port .\ ignore) (display "\#\texttildelow" port)
  (list (\%record->list r)))}.

\texttt{\%to-equal?}\ method to compare a \emph{record} to any value
using \texttt{equal?}.  The method \emph{prediate} is called with the
\emph{record} and the comparison value as its arguments.  The default
method is \texttt{eq?}.

\texttt{\%to-eval} method to evaluate a \emph{record}.  \texttt{Eval}
evaluates a \emph{record} by returning the value of calling the
\emph{record}'s \texttt{\%to-eval} method with the \emph{record} as
the argument.  The default method is \texttt{(lambda (x) x)}.

\texttt{\%to-write} method to \texttt{write} a \emph{record}.  When
\texttt{write} encounters a record, it calls the record's
\texttt{\%to-write} method with the following arguments: the record,
the output port, the number of spaces to indent (or \texttt{\#f}), the
number of levels to print (or \texttt{\#f}), the length of lists,
vectors, or records to print (or \texttt{\#f}), and a list of pairs,
vectors, and records already seen (or \texttt{\#f}).  The method
returns either \texttt{\#f} indicating no further action is to be
taken, or a pair indicating that the car of the pair is to be output.
For example, if \texttt{\%record-prefix-char} is
\texttt{\#\textbackslash\texttildelow}, the method could be:
\texttt{(lambda (r port .\ ignore) (display "\#\texttildelow" port)
  (list (\%record->list r)))}.

\texttt{'}\emph{expression} abbreviation for (\texttt{quote}
\emph{expression}).  \RRRRRS~7, 16.

(\texttt{*} \emph{number} ...)\ returns the product of its arguments.
\RRRRRS~21.

\texttt{*args*} arguments of the \emph{procedure} when a breakpoint
has been hit.  The value of this symbol will be used as the arguments
when the user continues from the breakpoint.  See \texttt{bpt},
\texttt{proceed}.

\texttt{*bpt-env*} list of environments when a breakpoint is
encountered in an embedded \StoC\ system.

\texttt{*error-env*} list of environments when an error occurs in an
embedded \StoC\ system.

\texttt{*error-handler*} the error handling \emph{procedure}.  See
\texttt{error}.

\texttt{*frozen-objects*} list of objects that are never moved by the
garbage collector.  Scheme programs can use this to ``lock down''
objects in memory before passing them to programs written in other
languages.

\texttt{*obarray*} is a vector of lists of symbols.  It is used by
\texttt{read} to assure that symbols written and then read back in are
\texttt{eqv?}.  See \emph{interned}, \RRRRRS~18.

\texttt{*result*} result of the \emph{procedure} when a breakpoint has
been hit.  The value of this symbol be returned as the value of the
\emph{procedure} after the user continues from the breakpoint.  See
\texttt{bpt}, \texttt{proceed}.

\texttt{*scheme2c-result*} normal result of computation in an embedded
\StoC\ system.

\texttt{`}\emph{back-quote-template} abbreviation for
(\texttt{quasiquote} \emph{back-quote template}). \RRRRRS~11.

\texttt{(} used to group and notate lists.  \RRRRRS~5.

\texttt{()} the empty list.  \RRRRRS~15.

\texttt{)} used to group and notate lists.  \RRRRRS~5.

(\texttt{+} \emph{number} ...)\ returns the sum of its arguments.
\RRRRRS~21.

\texttt{,}\emph{expression} abbreviation for (\texttt{unquote}
\emph{expression}) that causes the expression to be replaced by its
value in the \emph{back-quote-template}.  \RRRRRS~11.

\texttt{,@}\emph{expression} abbreviation for
(\texttt{unquote-splicing} \emph{expression}) that causes the
expression to be evaluated and ``spliced'' into the
\emph{back-quote-template}. \RRRRRS~11.

(\texttt{-} \emph{number} \emph{number} ...)\ with two or more
arguments, this returns the difference of its arguments, associating
to the left.  With one argument it returns the additive inverse of the
argument.  \RRRRRS~21.

\texttt{-C} command line flag to \texttt{s2cc} that will cause the
compiler to compile the Scheme files \emph{source}\texttt{.sc} to C
source in \emph{source}\texttt{.c}.  No further processing is
performed.

\texttt{-I} \emph{directory} command line flag to \texttt{s2cc} to
supply a directory to be searched by \texttt{include} when it is
looking for a source file.  When multiple flags are supplied, the
directories are searched in the order that the flags are specified.

\texttt{-LIBDIR} \emph{directory} command line flag to \texttt{s2cc} to
supply the \emph{directory} containing the files: predef.sc,
objects.h, libs2c.a, and optionally libs2c\_p.a.

\texttt{-Ob} command line flag to \texttt{s2cc} that controls bounds
checking.  When it is supplied to the compiler, no bounds checking
code for \emph{vector} or \emph{string} accesses will be generated.
Supplying this flag is equivalent to supplying the flags \texttt{-f
  '*bounds-check*' '\#f'}.

\texttt{-Og} command line flag to \texttt{s2cc} that controls the
generation of stack-trace debugging code. When it is supplied to the
compiler, stack-trace code will not be generated.

\texttt{-On} command line flag to \texttt{s2cc} that controls number
representation. When it is supplied to the compiler, all numbers will
be assumed to be \emph{fixed} integers.  Supplying this flag is
equivalent to supplying the flags \texttt{-f '*fixed-only*' '\#t'}.

\texttt{-Ot} command line flag to \texttt{s2cc} that controls type
error checking. When it is supplied, no error checking code will be
generated. Supplying this flag is equivalent to supplying the flags
\texttt{-f '*type-check*' '\#f'}.

\texttt{-e} command line flag to \texttt{s2ci}.  When it is supplied,
all text read on the standard input file will be echoed on the
standard output file.

\texttt{-emacs} command line flag to \texttt{s2ci}.  When supplied, the
interpreter assumes that it is being run by GNU emacs.

\texttt{-i} command line flag to \texttt{s2cc} that will combine the
source and object files into a Scheme interpreter.  Module names for
files other than Scheme source files must be supplied using the
\texttt{-m} command line flag.

\texttt{-log} command line flag to \texttt{s2cc} to log information
internal to the compiler.  Each type of compiler information is
denoted by one of the flags: \texttt{-source}, \texttt{-macro},
\texttt{-expand}, \texttt{-closed}, \texttt{-transform},
\texttt{-lambda}, \texttt{-tree}, \texttt{-lap}, \texttt{-peep}.  The
flag \texttt{-log} is equivalent to specifying the flags:
\texttt{-source}, \texttt{-macro}, \texttt{-expand}, \texttt{-closed},
\texttt{-transform}, \texttt{-lambda}, and \texttt{-tree}.

\texttt{-m} \emph{module} command line flag to \texttt{s2cc} to specify
the name of a module that must be initialized by calling the procedure
\emph{module\_\_init}.  Note that the Scheme compiler will downshift
the alphabetic characters in module names supplied in the
\texttt{module} directive.  Modules are initialized in the order that
the \texttt{-m} command flags are specified.

\texttt{-nh} command line flag to \texttt{s2ci}.  When it is supplied,
the interpreter version header will not be printed on the standard
output file.

\texttt{-np} command line flag to \texttt{s2ci}.  When it is supplied,
prompts for input from standard input will not be printed on standard
output.

\texttt{-q} command line flag to \texttt{s2ci}.  When it is supplied,
the result of each expression evaluation will not be printed on
standard output.

\texttt{-pg} command line flag to \texttt{s2cc} that will cause it to
produce profiled code for run-time measurement using \emph{gprof}.
The profiled library will be used in lieu of the standard Scheme
library.

\texttt{-scgc} \emph{flag} command line flag to any Scheme program
that controls the reporting of garbage collection statistics.  If
\emph{flag} is set to 1, then garbage collection statistics will be
printed on stderr.  This flag will override \texttt{S2CGCINFO}.

\texttt{-sch} \emph{integer} command line flag to any Scheme program
to set the initial heap size in megabytes.  If it is not supplied, and
the \texttt{S2CHEAP} environment variable was not set, and the program
did not have a default, then the implementation dependent default is
used.  This flag will override \texttt{S2CHEAP}.

\texttt{-scl} \emph{integer} command line flag to any Scheme program
to set the full collection limit.  When more than this percent of the
heap is allocated following a generational garbage collection, then a
full garbage collection will be done.  The default value is 40.  This
flag will override \texttt{S2CLIMIT}.

\texttt{-scm} \emph{symbol} command line flag to any Scheme program to
cause execution to start at the procedure that is the value of
\emph{symbol}, rather than at the main program.  Note that the Scheme
\texttt{read} procedure typically upshifts alphabetic characters.
Thus, to start execution in the Scheme interpreter, one would enter
\texttt{-scm READ-EVAL-PRINT} on the command line.

\texttt{-scmh} \emph{integer} command line flag to any Scheme program
to set the maximum heap size in megabytes.  If it is not supplied, and
the \texttt{S2CMAXHEAP} environment variable was not set, then the
maximum heap size is five times the initial heap size. This flag will
override \texttt{S2CMAXHEAP}.

\texttt{.}\ denotes a dotted-pair: (\emph{obj}
\texttt{.}\ \emph{obj}).  \RRRRRS~15.

\texttt{.sc} file name extension for \StoC\ source files.

(\texttt{/} \emph{number} ...)\ with two or more arguments, this
returns the quotient of its arguments, associating to the left.  With
one argument it returns the multiplicative inverse of the argument.
\RRRRRS~21.

\texttt{;} indicates the start of a comment.  The comment continues
until the end of the line.  \RRRRRS~5.

(\texttt{<} \emph{number} \emph{number} \emph{number}
...)\ \emph{predicate} that returns \texttt{\#t} when the arguments
are monotonically increasing.  \RRRRRS~21.

(\texttt{<=} \emph{number} \emph{number} \emph{number}
...)\ \emph{predicate} that returns \texttt{\#t} when the arguments
are monotonically nondecreasing.  \RRRRRS~21.

(\texttt{=} \emph{number} \emph{number} \emph{number}
...)\ \emph{predicate} that returns \texttt{\#t} when the arguments
are equal.  \RRRRRS~21.

\texttt{=>} used in a \texttt{cond} conditional clause.  \RRRRRS~9.

(\texttt{>} \emph{number} \emph{number} \emph{number}
...)\ \emph{predicate} that returns \texttt{\#t} when the arguments
are monotonically decreasing.  \RRRRRS~21.

(\texttt{>=} \emph{number} \emph{number} \emph{number}
...)\ \emph{predicate} that returns \texttt{\#t} when the arguments
are monotonically nonincreasing. \RRRRRS~21.

\texttt{\textbackslash} tells \texttt{read} to treat the character
that follows it as a letter when reading a symbol.  If the character
is a lower-case alphabetic character, it will not be upshifted.
\RRRRRS~18.

\texttt{\textbackslash"} represents a \texttt{"} inside a string
constant.  \RRRRRS~25.

\texttt{\textbackslash\textbackslash} represents a
\texttt{\textbackslash} inside a string constant.  \RRRRRS~25.

(\texttt{abs} \emph{number}) returns the magnitude of its argument.
\RRRRRS~21.

(\texttt{acos} \emph{number}) returns the arccosine of its argument.
\RRRRRS~23.

\texttt{after-collect} is a variable in the top level environment.
Following each garbage collection, if its value is not \texttt{\#f},
then it is assumed to be a procedure and is called with three
arguments: the heap size in bytes, the currently allocated storage in
bytes, and the allocation percentage that will cause a full garbage
collection.  The value returned by the procedure is ignored.

\emph{alist} a list of \emph{pairs}.  \RRRRRS~17.

(\texttt{and} \emph{expression} ...)\ \emph{syntax} for a conditional
expression.  \RRRRRS~9.

(\texttt{append} \emph{list} ...)\ returns a list consisting of the
elements of the first \emph{list} followed by the elements of the
other \emph{lists}.  \RRRRRS~17.

(\texttt{apply} \emph{procedure} \emph{arg-list}) calls the
\emph{procedure} with the elements of \emph{arg-list} as the actual
arguments.  \RRRRRS~27.

(\texttt{apply} \emph{procedure} \emph{obj} ...\ \emph{arg-list})
calls the \emph{procedure} with the list (\texttt{append}
(\texttt{list} \emph{obj} ...)\ \emph{arg-list}) as the actual
arguments.  \RRRRRS~27.

(\texttt{asin} \emph{number}) returns the arcsine of its argument.
\RRRRRS~23.

(\texttt{assoc} \emph{obj} \emph{alist}) finds the first \emph{pair}
in \emph{alist} whose \texttt{car} field is \texttt{equal?}\ to
\emph{obj}.  If no such \emph{pair} exists, then \texttt{\#f} is
returned.  \RRRRRS~17.

(\texttt{assq} \emph{obj} \emph{alist}) finds the first \emph{pair} in
\emph{alist} whose \texttt{car} field is \texttt{eq?}\ to \emph{obj}.
If no such \emph{pair} exists, then \texttt{\#f} is returned.
\RRRRRS~17.

(\texttt{assv} \emph{obj} \emph{alist}) finds the first \emph{pair} in
\emph{alist} whose \texttt{car} field is \texttt{eqv?}\ to \emph{obj}.
If no such \emph{pair} exists, then \texttt{\#f} is returned.
\RRRRRS~17.

(\texttt{atan} \emph{number}) returns the arctangent of its argument.
\RRRRRS~23.

(\texttt{atan} \emph{number} \emph{number}) returns the arctangent of
its arguments.  \RRRRRS~23.

(\texttt{backtrace}) displays the call stack where a breakpoint
occurred.

\emph{back-quote-template} list or vector structure that may contain
\texttt{,}\emph{expression} and \texttt{,@}\emph{expression} forms.
\RRRRRS~11.

(\texttt{begin} \emph{expression} ...)\ \emph{syntax} where
\emph{expression}'s are evaluated left to right and the value of the
last \emph{expression} is returned. \RRRRRS~10.

\emph{bindings} a \emph{list} whose elements are of the form:
(\emph{symbol} \emph{expression}), where the \emph{expression} is the
initial value to place in the location bound to the
\emph{symbol}. \RRRRRS~10.

(\texttt{bit-and} \emph{number} ...)\ returns an unsigned number
representing the bitwise and of its 32-bit arguments.

(\texttt{bit-lsh} \emph{number} \emph{integer}) returns an unsigned
number representing the 32-bit value \emph{number} shifted left
\emph{integer} bits.

(\texttt{bit-not} \emph{number} ...)\ returns an unsigned number
representing the bitwise not of its 32-bit argument.

(\texttt{bit-or} \emph{number} ...)\ returns an unsigned number
representing the bitwise inclusive or of its 32-bit arguments.

(\texttt{bit-rsh} \emph{number} \emph{integer}) returns an unsigned
number representing the 32-bit value \emph{number} shifted right
\emph{integer} bits.

(\texttt{bit-xor} \emph{number} ...)\ returns an unsigned number
representing the bitwise exclusive or of its 32-bit arguments.

\emph{body} one or more \emph{expressions} that are be executed in
sequence. \RRRRRS~10.

(\texttt{boolean?}\ \emph{expression}) \emph{predicate} that returns
\texttt{\#t} if \emph{expression} is \texttt{\#t} or \texttt{\#f}.
\RRRRRS~13.

(\texttt{bpt}) \emph{syntax} to return a list of the procedures that
have been breakpointed.

(\texttt{bpt} \emph{symbol}) \emph{syntax} to set a breakpoint on the
\emph{procedure} that is the value of \emph{symbol}.  Each entry and
exit of the \emph{procedure} will provide the user with an opportunity
to examine and alter the current state of the computation.  For
interactive \StoC\ systems, the computation is continued by entering
control-D.  The computation may be terminated and a return made to the
top level of the interpreter by entering \texttt{(top-level)}.  In
embedded \StoC\ systems, \texttt{(proceed)} is used to continue the
computation, and the computation is abandoned by evaluating
\texttt{(reset-error)}.See \texttt{*args*}, \texttt{*result*},
\texttt{top-level}, \texttt{unbpt}.

(\texttt{bpt} \emph{symbol} \emph{procedure}) \emph{syntax} to set a
conditional breakpoint on the \emph{procedure} that is the value of
\emph{symbol}.  A breakpoint occurs when (\texttt{apply}
\emph{procedure} \emph{arguments}) returns a true value.

(\texttt{c-byte-ref} \emph{c-pointer} \emph{integer}) returns the byte
at the \emph{integer} byte of \emph{c-pointer} as a \emph{number}.

(\texttt{c-byte-set!}\ \emph{c-pointer} \emph{integer} \emph{number})
sets the byte at the \emph{integer} byte of \emph{c-pointer} to
\emph{number} and returns \emph{number} as its value.

(\texttt{c-double-ref} \emph{c-pointer} \emph{integer}) returns the
double at the \emph{integer} byte of \emph{c-pointer} as a
\emph{number}.

(\texttt{c-double-set!}\ \emph{c-pointer} \emph{integer}
\emph{number}) sets the double at the \emph{integer} byte of
\emph{c-pointer} to \emph{number} and returns \emph{number} as its
value.

(\texttt{c-float-ref} \emph{c-pointer} \emph{integer}) returns the
float at the \emph{integer} byte of \emph{c-pointer} as a
\emph{number}.

(\texttt{c-float-set!}\ \emph{c-pointer} \emph{integer} \emph{number})
sets the float at the \emph{integer} byte of \emph{c-pointer} to
\emph{number} and returns \emph{number} as its value.

(\texttt{c-int-ref} \emph{c-pointer} \emph{integer}) returns the int
at the \emph{integer} byte of \emph{c-pointer} as a \emph{number}.

(\texttt{c-int-set!}\ \emph{c-pointer} \emph{integer} \emph{number})
sets the int at the \emph{integer} byte of \emph{c-pointer} to
\emph{number} and returns \emph{number} as its value.

(\texttt{c-longint-ref} \emph{c-pointer} \emph{integer}) returns the
long int at the \emph{integer} byte of \emph{c-pointer} as a
\emph{number}.

(\texttt{c-longint-set!}\ \emph{c-pointer} \emph{integer}
\emph{number}) sets the long int at the \emph{integer} byte of
\emph{c-pointer} to \emph{number} and returns \emph{number} as its
value.

(\texttt{c-longunsigned-ref} \emph{c-pointer} \emph{integer}) returns
the unsigned long at the \emph{integer} byte of \emph{c-pointer} as a
\emph{number}.

(\texttt{c-longunsigned-set!}\ \emph{c-pointer} \emph{integer}
\emph{number}) sets the unsigned long at the \emph{integer} byte of
\emph{c-pointer} to \emph{number} and returns \emph{number} as its
value.

\emph{c-pointer} a \emph{number} that is the address of a structure
outside the Scheme heap, or a \emph{string} that is a C-structure
within the Scheme heap.

(\texttt{c-s2cuint-ref} \emph{c-pointer} \emph{integer}) returns the
S2CUINT at the \emph{integer} byte of \emph{c-pointer} as a
\emph{number}.

(\texttt{c-s2cuint-set!}\ \emph{c-pointer} \emph{integer}
\emph{number}) sets the S2CUINT at the \emph{integer} byte of
\emph{c-pointer} to \emph{number} and returns \emph{number} as its
value.

(\texttt{c-shortint-ref} \emph{c-pointer} \emph{integer}) returns the
short int at the \emph{integer} byte of \emph{c-pointer} as a
\emph{number}.

(\texttt{c-shortint-set!}\ \emph{c-pointer} \emph{integer}
\emph{number}) sets the short int at the \emph{integer} byte of
\emph{c-pointer} to \emph{number} and returns \emph{number} as its
value.

(\texttt{c-shortunsigned-ref} \emph{c-pointer} \emph{integer}) returns
the unsigned short at the \emph{integer} byte of \emph{c-pointer} as a
\emph{number}.

(\texttt{c-shortunsigned-set!}\ \emph{c-pointer} \emph{integer}
\emph{number}) sets the unsigned short at the \emph{integer} byte of
\emph{c-pointer} to \emph{number} and returns \emph{number} as its
value.

\texttt{c-sizeof-double} size (in bytes) of the C type double.

\texttt{c-sizeof-float} size (in bytes) of the C type float.

\texttt{c-sizeof-int} size (in bytes) of the C type int.

\texttt{c-sizeof-long} size (in bytes) of the C type long.

\texttt{c-sizeof-s2cuint} size (in bytes) of the C type S2CUINT that
is defined by \StoC\ to be an unsigned integer the same size as a
pointer.

\texttt{c-sizeof-short} size (in bytes) of the C type short.

\texttt{c-sizeof-tscp} size (in bytes) of the C type TSCP that is
defined by \StoC\ to represent tagged Scheme pointers.

(\texttt{c-string->string} \emph{c-pointer}) returns a Scheme
\emph{string} that is a copy of the null-terminated string
\emph{c-pointer}.

(\texttt{c-tscp-ref} \emph{c-pointer} \emph{integer}) returns the TSCP
at the \emph{integer} byte of \emph{c-pointer}.

(\texttt{c-tscp-set!}\ \emph{c-pointer} \emph{integer}
\emph{expression}) sets the TSCP at the \emph{integer} byte of
\emph{c-pointer} to \emph{expression} and returns \emph{expression} as
its value.

(\texttt{c-unsigned-ref} \emph{c-pointer} \emph{integer}) returns the
unsigned at the \emph{integer} byte of \emph{c-pointer} as a
\emph{number}.

(\texttt{c-unsigned-set!}\ \emph{c-pointer} \emph{integer}
\emph{number}) sets the unsigned at the \emph{integer} byte of
\emph{c-pointer} to \emph{number} and returns \emph{number} as its
value.

\emph{c-type} \emph{syntax} for declaring the type of a non-Scheme
procedure, procedure argument, or global.  The allowed types are:
\texttt{pointer}, \texttt{array}, \texttt{char}, \texttt{int},
\texttt{shortint}, \texttt{longint}, \texttt{unsigned},
\texttt{shortunsigned}, \texttt{longunsigned}, \texttt{float},
\texttt{double}, \texttt{tscp}, or \texttt{void}.

(\texttt{car} \emph{pair}) returns the contents of the \texttt{car}
field of the \emph{pair}.  \RRRRRS~16.

(\texttt{caar} \emph{pair}) returns (\texttt{car} (\texttt{car}
\emph{pair})).  \RRRRRS~16.

(\texttt{ca...r} \emph{pair}) compositions of \texttt{car} and
\texttt{cdr}.  \RRRRRS~16.

(\texttt{call-with-current-continuation} \emph{procedure}) calls
\emph{procedure} with the current continuation as its argument.
\RRRRRS~28.

(\texttt{call-with-input-file} \emph{string} \emph{procedure}) calls
\emph{procedure} with the \emph{port} that is the result of opening
the file \emph{string} for input.  \RRRRRS~29.

(\texttt{call-with-output-file} \emph{string} \emph{procedure}) calls
\emph{procedure} with the \emph{port} that is the result of opening
the file \emph{string} for output.  \RRRRRS~29.

(\texttt{case} \emph{key} \emph{clause} \emph{clause}
...)\ \emph{syntax} for a conditional expression where \emph{key} is
any expression, and each \emph{clause} is of the form ((\emph{datum}
...)\ \emph{expression} \emph{expression} ...).  The last clause may
be an ``else clause'' of the form (\texttt{else} \emph{expression}
\emph{expression} ...).  \RRRRRS~9.

(\texttt{catch-error} \emph{procedure}) calls \emph{procedure} with no
arguments.  If an error occurs while executing \emph{procedure},
return a string containing the error message.  Otherwise return a
\emph{pair} whose \texttt{car} contains the procedure's value.

(\texttt{cdr} \emph{pair}) returns the contents of the \texttt{cdr}
field of the \emph{pair}.  \RRRRRS~16.

(\texttt{cd...r} \emph{pair}) compositions of \texttt{car} and
\texttt{cdr}.  \RRRRRS~16.

(\texttt{cddddr} \emph{pair}) returns (\texttt{cdr} (\texttt{cdr}
(\texttt{cdr} (\texttt{cdr} \emph{pair})))).  \RRRRRS~16.

(\texttt{ceiling} \emph{number}) returns the smallest integer that is
not smaller than its arguments.  \RRRRRS~22.

\texttt{char} \emph{syntax} for declaring a non-Scheme procedure,
procedure argument, or global variable as the C type \texttt{char}.
When a \texttt{char} value must be supplied, an expression of type
\emph{character} must be supplied.  When a \texttt{char} value is
returned, a value of type \emph{character} will be returned.

(\texttt{char->integer} \emph{character}) returns an \emph{integer}
whose value is the ASCII character code of \emph{character}.
\RRRRRS~25.

(\texttt{char-alphabetic?}\ \emph{character}) \emph{predicate} that
returns \texttt{\#t} when \emph{character} is alphabetic.  \RRRRRS~25.

(\texttt{char-ci<=?}\ \emph{character} \emph{character})
\emph{predicate} that returns \texttt{\#t} when the first
\emph{character} is less than or equal to the second \emph{character}.
Upper case and lower case letters are treated as though they were the
same character.  \RRRRRS~25.

(\texttt{char-ci<?}\ \emph{character} \emph{character})
\emph{predicate} that returns \texttt{\#t} when the first
\emph{character} is less than the second \emph{character}.  Upper case
and lower case letters are treated as though they were the same
character. \RRRRRS~25.

(\texttt{char-ci=?}\ \emph{character} \emph{character})
\emph{predicate} that returns \texttt{\#t} when the first
\emph{character} is equal to the second \emph{character}. Upper case
and lower case letters are treated as though they were the same
character.  \RRRRRS~25.

(\texttt{char-ci>=?}\ \emph{character} \emph{character})
\emph{predicate} that returns \texttt{\#t} when the first
\emph{character} is greater than or equal to the second
\emph{character}. Upper case and lower case letters are treated as
though they were the same character. \RRRRRS~25.

(\texttt{char-ci>?}\ \emph{character} \emph{character})
\emph{predicate} that returns \texttt{\#t} when the first
\emph{character} is greater than the second \emph{character}. Upper
case and lower case letters are treated as though they were the same
character. \RRRRRS~25.

(\texttt{char-downcase} \emph{character}) returns the lower case value
of \emph{character}.  \RRRRRS~25.

(\texttt{char-lower-case?}\ \emph{letter}) \emph{predicate} that
returns \texttt{\#t} when \emph{letter} is lower-case.  \RRRRRS~25.

(\texttt{char-numeric?}\ \emph{character}) \emph{predicate} that
returns \texttt{\#t} when \emph{character} is numeric.  \RRRRRS~25.

(\texttt{char-ready?}\ \emph{optional-input-port}) \emph{predicate}
that returns \texttt{\#t} when a character is ready on the
\emph{optional-input-port}.  \RRRRRS~30.

(\texttt{char-upcase} \emph{character}) returns the upper case value
of the \emph{character}.  \RRRRRS~25.

(\texttt{char-upper-case?}\ \emph{letter}) \emph{predicate} that
returns \texttt{\#t} when \emph{letter} is upper-case.  \RRRRRS~25.

(\texttt{char-whitespace?}\ \emph{character}) \emph{predicate} that
returns \texttt{\#t} when \emph{character} is a whitespace character.
\RRRRRS~25.

(\texttt{char<=?}\ \emph{character} \emph{character}) \emph{predicate}
that returns \texttt{\#t} when the first \emph{character} is less than
or equal to the second \emph{character}.  \RRRRRS~24.

(\texttt{char<?}\ \emph{character} \emph{character}) \emph{predicate}
that returns \texttt{\#t} when the first \emph{character} is less than
the second \emph{character}.  \RRRRRS~24.

(\texttt{char=?}\ \emph{character} \emph{character}) \emph{predicate}
that returns \texttt{\#t} when the first \emph{character} is equal to
the second \emph{character}.  \RRRRRS~24.

(\texttt{char>=?}\ \emph{character} \emph{character}) \emph{predicate}
that returns \texttt{\#t} when the first \emph{character} is greater
than or equal to the second \emph{character}.  \RRRRRS~24.

(\texttt{char>?}\ \emph{character} \emph{character}) \emph{predicate}
that returns \texttt{\#t} when the first \emph{character} is greater
than the second \emph{character}.  \RRRRRS~24.

(\texttt{char?}\ \emph{expression}) \emph{predicate} that returns
\texttt{\#t} when \emph{expression} is a \emph{character}.
\RRRRRS~24.

\emph{character} Scheme object that represents printed characters.
See \texttt{\#\textbackslash}\emph{character},
\texttt{\#\textbackslash}\emph{character-name}, \RRRRRS~24.

(\texttt{close-input-port} \emph{input-port}) closes the file
associated with \emph{input-port}. \RRRRRS~30.

(\texttt{close-output-port} \emph{output-port}) closes the file
associated with \emph{output-port}.  \RRRRRS~30.

(\texttt{close-port} \emph{port}) closes the file associated with
\emph{port}.

(\texttt{collect}) invokes the garbage collector to perform a
generational collection.  Normally, garbage collection is invoked
automatically by the Scheme system.

(\texttt{collect-all}) invokes the garbage collector to perform a full
collection.  Normally, garbage collection is invoked automatically by
the Scheme system.

(\texttt{collect-info}) returns a \emph{list} containing information
about heap and prcessor usage.  The items in the list (and their
position) are: number of bytes currently allocated (0), current heap
size in bytes (1), application processor seconds (2), garbage
collection processor seconds (3), maximum heap size in bytes (4), full
collection limit percent (5).

\emph{complex number} complex numbers are not supported in \StoC.
\RRRRRS~18.

(\texttt{complex?}\ \emph{expression}) \emph{predicate} that returns
\texttt{\#t} when \emph{expression} is a \emph{complex number}.  All
\StoC\ \emph{numbers} are complex. \RRRRRS~20.

(\texttt{cond} \emph{clause} \emph{clause} ...)\ \emph{syntax} for a
conditional expression where each \emph{clause} is of the form
(\emph{test} \emph{expression} ...)\ or (\emph{test} \texttt{=>}
\emph{procedure}).  The last \emph{clause} may be of the form
(\texttt{else} \emph{expression} \emph{expression} ...).  \RRRRRS~9.

(\texttt{cons} \emph{expression}$_1$ \emph{expression}$_2$) returns a
newly allocated \emph{pair} that has \emph{expression}$_1$ as its
\texttt{car}, and \emph{expression}$_2$ as its
\texttt{cdr}. \RRRRRS~16.

(\texttt{cons*} \emph{expression} \emph{expression} ...)\ returns an
object formed by consing the \emph{expressions} together from right to
left. If only one \emph{expression} is supplied, then that
\emph{expression} is returned.

(\texttt{cos} \emph{number}) returns the cosine of its argument.
\RRRRRS~23.

(\texttt{current-input-port}) returns the current default input
\emph{port}.  \RRRRRS~30.

(\texttt{current-output-port}) returns the current default output
\emph{port}.  \RRRRRS~30.

\texttt{debug-output-port} \emph{port} used for interactive debugging
output.  The default value is the same as \texttt{stderr-port}.

(\texttt{define} \emph{symbol} \emph{expression}) \emph{syntax} that
defines the value of \emph{expression} as the value of either a
top-level symbol or a local variable.  \RRRRRS~12.

(\texttt{define} (\emph{symbol} \emph{formals}) \emph{body})
\emph{syntax} that defines a \emph{procedure} that is either the value
of a top-level symbol or a local variable. \RRRRRS~12.

(\texttt{define} (\emph{symbol} \texttt{.}\ \emph{formal})
\emph{body}) \emph{syntax} that defines a \emph{procedure} that is
either the value of a top-level symbol or a local
variable. \RRRRRS~12.

(\texttt{define-c-external} \emph{symbol} \emph{c-type} \emph{string})
\emph{syntax} for a compiler declaration that defines \emph{symbol} as
a non-Scheme global variable with the name \emph{string} and the type
\emph{c-type}.

(\texttt{define-c-external} (\emph{symbol}
\emph{c-type}$_1$...)\ \emph{c-type}$_2$ \emph{string}) \emph{syntax}
for a compiler declaration that defines \emph{symbol} as a non-Scheme
procedure with arguments of the type specified in the list
\emph{c-type}$_1$.  The procedure's name is \emph{string} and it
returns a value of type \emph{c-type}$_2$.

(\texttt{define-c-external} (\emph{symbol} \emph{c-type}$_1$...\ .
\emph{c-type}$_2$) \emph{c-type}$_3$ \emph{string}) \emph{syntax} for
a compiler declaration that defines \emph{symbol} as a non-Scheme
procedure that takes a variable number of arguments.  The types of the
initial arguments are specified by the list \emph{c-type}$_1$.  Any
additional arguments must be of the type \emph{c-type}$_2$.  The
procedure's name is \emph{string} and it returns a value of type
\emph{c-type}$_3$.

(\texttt{define-constant} \emph{symbol} \emph{expression})
\emph{syntax} that defines a macro that replaces all occurences of
\emph{symbol} with the value of \emph{expression}, evaluated at the
time of the definition.

(\texttt{define-external} \emph{symbol}$_1$ \emph{symbol}$_2$)
\emph{syntax} for a compiler declaration that \emph{symbol}$_1$ is
defined in \emph{module} \emph{symbol}$_2$.

(\texttt{define-external} \emph{symbol} \texttt{TOP-LEVEL})
\emph{syntax} for a compiler declaration that \emph{symbol} is a
top-level symbol.  Its value is to be found via the
\texttt{*obarray*}.

(\texttt{define-external} \emph{symbol}$_1$ \texttt{TOP-LEVEL}
\emph{symbol}$_2$) \emph{syntax} for a compiler declaration that
\emph{symbol}$_1$ is a top-level symbol that is known to be defined in
\emph{module} \emph{symbol}$_2$.  Its value is to be found via the
\texttt{*obarray*}.

(\texttt{define-external} \emph{symbol} \texttt{""} \emph{string})
\emph{syntax} for a compiler declaration that \emph{symbol} has the
external name \emph{string}.

(\texttt{define-external} \emph{symbol} \emph{string}$_1$
\emph{string}$_2$) \emph{syntax} for a compiler declaration that
\emph{symbol} is in the \emph{module} \emph{string}$_1$ and has the
external name \emph{string}$_1$\_\emph{string}$_2$.

(\texttt{define-external} (\emph{symbol}$_1$ \emph{formals})
\emph{symbol}$_2$) \emph{syntax} for a compiler declaration that
\emph{symbol}$_1$ is a Scheme \emph{procedure} defined in
\emph{module} \emph{symbol}$_2$.

(\texttt{define-external} (\emph{symbol}$_1$
\texttt{.}\ \emph{formal}) \emph{symbol}$_2$) \emph{syntax} for a
compiler declaration that \emph{symbol}$_1$ is a Scheme
\emph{procedure} defined in \emph{module} \emph{symbol}$_2$.

(\texttt{define-external} (\emph{symbol} \emph{formals}) \texttt{""}
\emph{string}) \emph{syntax} for a compiler declaration that
\emph{symbol} is a \emph{procedure} that has the external name
\emph{string}.

(\texttt{define-external} (\emph{symbol} \texttt{.}\ \emph{formal})
\texttt{""} \emph{string}) \emph{syntax} for a compiler declaration
that \emph{symbol} is a \emph{procedure} that takes a variable number
of arguments and has the external name \emph{string}.

(\texttt{define-external} (\emph{symbol} \emph{formals})
\emph{string}$_1$ \emph{string}$_2$) \emph{syntax} for a compiler
declaration that \emph{symbol} is a \emph{procedure} in the
\emph{module} \emph{string}$_1$ that has the external name
\emph{string}$_1$\_\emph{string}$_2$.

(\texttt{define-external} (\emph{symbol} \texttt{.}\ \emph{formal})
\emph{string}$_1$ \emph{string}$_2$) \emph{syntax} for a compiler
declaration that \emph{symbol} is a \emph{procedure} in the
\emph{module} \emph{string}$_1$ that has the external name
\emph{string}$_1$\_\emph{string}$_2$.

(\texttt{define-in-line} (\emph{symbol} \emph{formals}) \emph{body})
\emph{syntax} that defines a \emph{procedure} that is to be compiled
``in-line''.

(\texttt{define-in-line} (\emph{symbol} \texttt{.}\ \emph{formal})
\emph{body}) \emph{syntax} that defines a \emph{procedure} that is to
be compiled ``in-line''.

(\texttt{define-macro} \emph{symbol} (\texttt{lambda} (\emph{form
  expander}) \emph{expression} ...))\ \emph{syntax} that defines a
macro expansion procedure.  Macro expansion is done using the ideas
expressed in \emph{Expansion-Passing Style: Beyond Conventional
  Macros}, 1986 ACM Conference on Lisp and Functional Programming,
143-150.

(\texttt{define-system-file-task} \emph{file} \emph{idle-task}
\emph{file-task}) installs the \emph{idle-task} and \emph{file-task}
\emph{procedures} for system file number \emph{file}.  When a Scheme
program reads from a port and no characters are internally buffered,
the \emph{idle-task} for each system file is called.  Then, the
\emph{file-task} for each system file that has input pending is
called.  As long as no characters are available on the Scheme port,
the Scheme system will idle, calling the \emph{file-task} for each
system file as input becomes available.  A system file task is removed
by supplying \texttt{\#f} as the \emph{idle-task} and
\emph{file-task}.

(\texttt{delay} \emph{expression}) \emph{syntax} used together with
the procedure \texttt{force} to implement call by need.  \RRRRRS~11.

(\texttt{display} \emph{expression} \emph{optional-output-port})
writes a human-readable representation of \emph{expression} to
\emph{optional-output-port}.  \RRRRRS~31.

(\texttt{do} (\emph{var} ...)\ (\emph{test} \emph{expression}
...)\ \emph{command} ...)\ \emph{syntax} for an iteration construct.
Each \emph{var} defines a local variable and is of the form
(\emph{symbol} \emph{init} \emph{step}) or (\emph{symbol}
\emph{init}).  \RRRRRS~11.

\texttt{double} \emph{syntax} for declaring a non-Scheme procedure,
procedure argument, or global variable as the C type
\texttt{double}. When a \texttt{double} value must be supplied, an
expression of type \emph{number} must be supplied.  When a
\texttt{double} value is returned, a value of type \emph{number} is
returned.

(\texttt{echo} \emph{port}) turns off echoing on \emph{port}.

(\texttt{echo} \emph{port} \emph{output-port}) echos \emph{port} on
\emph{output-port}.  All characters read from or written to
\emph{port} are also written to \emph{output-port}.

\texttt{else} keyword in last \emph{clause} of \texttt{cond} or
\texttt{case} form.

\emph{environment} the set of all variable bindings in effect at some
point in the program.  \RRRRRS~6.

(\texttt{eof-object?}\ \emph{expression}) \emph{predicate} that
returns \texttt{\#t} if \emph{expression} is equal to the end of file
object.  \RRRRRS~30.

(\texttt{enable-system-file-tasks} \emph{flag}) enables (\emph{flag}
is \texttt{\#t}) or disables (\emph{flag} is \texttt{\#f}) system file
tasking and returns the previous system file tasking state.  When the
value of flag is the symbol \texttt{wait}, system file tasking is
enabled and the Scheme program is blocked until there are no system
file tasks.

(\texttt{eq?}\ \emph{expression}$_1$ \emph{expression}$_2$)
\emph{predicate} that is the finest test for equivalence between
\emph{expression}$_1$ and \emph{expression}$_2$.  \RRRRRS~15.

(\texttt{equal?}\ \emph{expression}$_1$ \emph{expression}$_2$)
\emph{predicate} that is the coarsest test for equivalence between
\emph{expression}$_1$ and \emph{expression}$_2$.  \RRRRRS~15.

(\texttt{eqv?}\ \emph{expression}$_1$ \emph{expression}$_2$)
\emph{predicate} that is the medium test for equivalence between
\emph{expression}$_1$ and \emph{expression}$_2$.  \RRRRRS~13.

(\texttt{error} \emph{symbol} \emph{format-template} \emph{expression}
...)\ reports an error.  The procedure name is \emph{symbol} and the
error message is produced by the \emph{format-template} and optional
\emph{expressions}.  The \emph{procedure} error is equivalent to
\texttt{(lambda x (apply *error-handler* x))}.  See
\texttt{*error-handler*}.

(\texttt{eval} \emph{expression}) evaluates \emph{expression}.  Any
macros in \emph{expression} are expanded before evaluation.

(\texttt{eval-when} \emph{list} \emph{expression} ...)\ \emph{syntax}
to evaluate \emph{expressions} when the current situation is in
\emph{list}.  When this form is evaluated by the Scheme interpreter
and \texttt{eval} is a member of the situation \emph{list}, then the
expressions will be evaluated.  When this form is evaluated by the
Scheme compiler and \texttt{compile} is a member of the situation
\emph{list}, then the expressions will be evaluated within the
compiler.  When this form is evaluated by the Scheme compiler, and
\texttt{load} is a member of the situation \emph{list}, then the
compiler will compile the form (\texttt{begin} \emph{expression}
...)).

(\texttt{even?}\ \emph{integer}) \emph{predicate} that returns
\texttt{\#t} if \emph{integer} is even.  \RRRRRS~21.

\emph{exact} \qquad \emph{fixed} numbers are exact, all other numbers
are not.  \RRRRRS~14.

(\texttt{exact->inexact} \emph{number}) returns the \emph{inexact}
representation of \emph{number}. \RRRRRS~23.

(\texttt{exact?}\ \emph{number}) \emph{predicate} that returns
\texttt{\#t} if \emph{number} is \emph{exact}.  \RRRRRS~21.

(\texttt{exit}) returns from the current \texttt{read-eval-print}
procedure.

(\texttt{exp} \emph{number}) returns exponential function of
\emph{number}.  \RRRRRS~22.

(\texttt{expand} \emph{expression}) returns the value of
\emph{expression} after all macro expansions.  See
\texttt{define-macro}.

(\texttt{expand-once} \emph{expression}) returns the value of
\emph{expression} after one macro expansion.  See
\texttt{define-macro}.

\emph{expression} a Scheme construct that returns a value.  \RRRRRS~7.

(\texttt{expt} \emph{number}$_1$ \emph{number}$_2$) returns
\emph{number}$_1$ raised to the power \emph{number}$_2$.  \RRRRRS~23.

\texttt{fix} \emph{format descriptor} for compatibility with \RRRRS.

\emph{fixed} \StoC\ internal representation for small \emph{integer}s.
A \emph{fixed} value is represented in a ``pointer size'' word with
two bits used by the tag.  With 32-bit pointers, this yields a maximum
value of $2^{29}-1$ or 536,870,911 and a minimum value of $-2^{29}$ or
$-$536,870,912.  With 64-bit pointers, this yields a maximum value of
$2^{61}-1$ or 2,305,843,009,213,693,951 and a minimum value of
$-2^{61}$ or $-$2,305,843,009,213,693,952.

(\texttt{fixed->float} \emph{fixed}) returns the \emph{float}
representation of \emph{fixed}.

(\texttt{fixed?}\ \emph{expression}) \emph{predicate} that returns
\texttt{\#t} when \emph{expression} is a \emph{fixed}.

\texttt{float} \emph{syntax} for declaring a non-Scheme procedure,
procedure argument, or global variable as the C type
\texttt{float}. When a \texttt{float} value must be supplied, an
expression of type \emph{number} must be supplied.  When a
\texttt{float} value is returned, a value of type \emph{number} is
returned.

\emph{float} \StoC\ internal floating point representation.  This is
typically 64-bits.

(\texttt{float->fixed} \emph{float}) returns the \emph{fixed}
\emph{number} that best represents the value of \emph{float}.

(\texttt{float?}\ \emph{expression}) \emph{predicate} that returns
\texttt{\#t} if \emph{expression} is a \emph{float} value.

(\texttt{floor} \emph{number}) returns the largest \emph{integer} not
larger than \emph{number}.  \RRRRRS~22.

(\texttt{flush-buffer} \emph{optional-output-port}) forces output of
all characters buffered in \emph{optional-output-port}.

(\texttt{for-each} \emph{procedure} \emph{list} \emph{list}
...)\ applies \emph{procedure} to each element of the \emph{lists} in
order.  \RRRRRS~28.

(\texttt{force} \emph{promise}) returns the forced value of a promise.
\RRRRRS~28.

\emph{formals} a \emph{symbol} or a \emph{list} of \emph{symbols} that
are the arguments.  \RRRRRS~8.

(\texttt{format} \texttt{\#f} \emph{format-template} \emph{expression}
...)\ returns a string that is the result of outputting the
\emph{expressions} according to the \emph{format-template}.

(\texttt{format} \emph{format-template} \emph{expression}
...)\ returns a string that is the result of outputting the
\emph{expressions} according to the \emph{format-template}.

(\texttt{format} \emph{output-port} \emph{format-template}
\emph{expression} ...)\ output the \emph{expressions} to
\emph{output-port} according to the \emph{format-template}.

(\texttt{format} \texttt{\#t} \emph{format-template} \emph{expression}
...)\ output the \emph{expressions} to the current output port
according to the \emph{format-template}.

\emph{format descriptor} a \emph{list} that describes the type of
output conversion to be done by \texttt{number->string}.  The
supported forms are (\texttt{int}), (\texttt{fix} \emph{integer}), and
(\texttt{s2ci} \emph{integer}).  \RRRRRS~21.

\emph{format-template} a \emph{string} consisting of format
descriptors and literal characters.  A format descriptor is
\texttt{\texttildelow} followed by some other character.  When one is
encountered, it is interpreted.  Literal characters are output as is.
See \texttt{\texttildelow{}a}, \texttt{\texttildelow{}A},
\texttt{\texttildelow{}c}, \texttt{\texttildelow{}C},
\texttt{\texttildelow{}s}, \texttt{\texttildelow{}S},
\texttt{\texttildelow\%}, \texttt{\texttildelow\texttildelow}.

(\texttt{gcd} \emph{number} ...)\ returns the greatest common divisor
of its arguments.  \RRRRRS~22.

(\texttt{get-output-string} \emph{string-output-port}) returns the
\emph{string} associated with \emph{string-output-port}.  The
\emph{string} associated with the \emph{string-output-port} is
initially set to \texttt{""}.

(\texttt{getprop} \emph{symbol} \emph{expression}) returns the value
that has the key \texttt{eq?}\ to \emph{expression} from
\emph{symbol's} property list.  If there is no value associated with
\emph{expression}, then \texttt{\#f} is returned.

(\texttt{getprop-all} \emph{symbol}) returns the \emph{symbol's}
property list.

(\texttt{implementation-information}) returns a list of string or
\texttt{\#f} values containing information about the Scheme
implementation.  The list is of the form (\emph{implementation-name}
\emph{version} \emph{machine} \emph{processor} \emph{operating-system}
\emph{filesystem} \emph{features} ...).

(\texttt{if} \emph{expression}$_1$ \emph{expression}$_2$)
\emph{syntax} for a conditional expression.  \RRRRRS~8.

(\texttt{if} \emph{expression}$_1$ \emph{expression}$_2$
\emph{expression}$_3$) \emph{syntax} for a conditional expression.
\RRRRRS~8.

(\texttt{include} \emph{string}) \emph{syntax} to include the contents
of the file \emph{string} at this point in the Scheme compilation.
Search directories may be specified by the \texttt{-I} command flag.

\emph{inexact} \qquad \emph{float} numbers are inexact.  \RRRRRS~14.

(\texttt{inexact->exact} \emph{number}) returns the \emph{exact}
representation of \emph{number}.  \RRRRRS~23.

(\texttt{inexact?}\ \emph{number}) \emph{predicate} that returns
\texttt{\#t} when \emph{number} is \emph{inexact}.  \RRRRRS~21.

\emph{input-port} Scheme object that can deliver characters on
command.  \RRRRRS~29.

(\texttt{input-port?}\ \emph{expression}) \emph{predicate} when
returns \texttt{\#t} when \emph{expression} is an \emph{input-port}.
\RRRRRS~29.

\texttt{int} \emph{syntax} for declaring a non-Scheme procedure,
procedure argument, or global variable as the C type \texttt{int}.
When a \texttt{int} value must be supplied, an expression of type
\emph{number} must be supplied.  When a \texttt{int} value is
returned, a value of type \emph{number} is returned.

\texttt{int} \emph{format descriptor} for compatibility with \RRRRS.

\emph{integer} integers are represented by both \emph{fixed} and
\emph{float} values.  \RRRRRS~18.

(\texttt{integer->char} \emph{integer}) returns the \emph{character}
whose ASCII code is equal to \emph{integer}.  \RRRRRS~25.

(\texttt{integer?}\ \emph{expression}) \emph{predicate} that returns
\texttt{\#t} when \emph{expression} is an \emph{integer}.  \RRRRRS~20.

\emph{interned} \qquad \emph{symbols} that are contained in
\texttt{*obarray*} are interned.

(\texttt{lambda} \emph{formals} \emph{body}) the ultimate imperative,
the ultimate declarative.  \RRRRRS~8.

(\texttt{last-pair} \emph{list}) returns the last \emph{pair} of
\emph{list}.

(\texttt{lcm} \emph{number} ...)\ returns the least common multiple of
its arguments.  \RRRRRS~22.

(\texttt{length} \emph{list}) returns the length of \emph{list}.
\RRRRRS~17.

(\texttt{let} \emph{bindings} \emph{body}) \emph{syntax} for a binding
construct that computes initial values before any bindings are done.
\RRRRRS~10.

(\texttt{let} \emph{symbol} \emph{bindings} \emph{body}) \emph{syntax}
for a general looping construct. \RRRRRS~11.

(\texttt{let*} \emph{bindings} \emph{body}) \emph{syntax} for a
binding construct that computes initial values and performs bindings
sequentially.  \RRRRRS~10.

(\texttt{letrec} \emph{bindings} \emph{body}) \emph{syntax} for a
binding construct that binds the variables before the initial values
are computed.  \RRRRRS~10.

\emph{letter} an alphabetic \emph{character}.  \RRRRRS~25.

\emph{list} the empty list, or a \emph{pair} whose \texttt{cdr} is a
\emph{list}.  \RRRRRS~16.

(\texttt{list} \emph{expression} ...)\ returns a \emph{list} of its
arguments.  \RRRRRS~17.

(\texttt{list?}\ \emph{expression}) \emph{predicate} that returns
\texttt{\#t} when \emph{expression} is a \emph{list}.  \RRRRRS~16.

(\texttt{list->string} \emph{list}) returns the string formed from the
\emph{characters} in \emph{list}.  \RRRRRS~26.

(\texttt{list->vector} \emph{list}) returns a \emph{vector} whose
elements are the members of \emph{list}.  \RRRRRS~27.

(\texttt{list-ref} \emph{list} \emph{integer}) returns the
\emph{integer} element of \emph{list}.  Elements are numbered starting
at 0.  \RRRRRS~17.

(\texttt{list-tail} \emph{list} \emph{integer}) returns the sublist of
\emph{list} obtained by omitting the first \emph{integer} elements.
\RRRRRS~17.

(\texttt{load} \emph{string}) loads the expressions in the file
\emph{string} into the Scheme interpreter.  The results of the
expressions are printed on the current output port.  \RRRRRS~31.

(\texttt{loade} \emph{string}) loads the expressions in the file
\emph{string} into the Scheme interpreter.  The contents of the file
and the results of the expressions are printed on the current output
port.

(\texttt{loadq} \emph{string}) loads the expressions in the file
\emph{string} into the Scheme interpreter.  The results of the
expressions are not printed.

(\texttt{log} \emph{number}) returns the natural logarithm of
\emph{number}.  \RRRRRS~22.

\texttt{longint} \emph{syntax} for declaring a non-Scheme procedure,
procedure argument, or global variable as the C type \texttt{long
  int}.  When a \texttt{long int} value must be supplied, an
expression of type \emph{number} must be supplied.  When a
\texttt{long int} value is returned, a value of type \emph{number} is
returned.

\texttt{longunsigned} \emph{syntax} for declaring a non-Scheme
procedure, procedure argument, or global variable as the C type
\texttt{long unsigned}. When a \texttt{long unsigned} value must be
supplied, an expression of type \emph{number} must be supplied.  When
a \texttt{long unsigned} value is returned, a value of type
\emph{number} is returned.

(\texttt{make-string} \emph{integer}) returns a string of length
\emph{integer} with unknown elements.  \RRRRRS~25.

(\texttt{make-string} \emph{integer} \emph{char}) returns a string of
length \emph{integer} with all elements initialized to \emph{char}.
\RRRRRS~25.

(\texttt{make-vector} \emph{integer}) returns a vector of length
\emph{integer} with unknown elements.  \RRRRRS~26.

(\texttt{make-vector} \emph{integer} \emph{expression}) returns a
vector of length \emph{integer} with all elements set to
\emph{expression}.  \RRRRRS~26.

(\texttt{map} \emph{procedure} \emph{list} \emph{list} ...)\ returns a
\emph{list} constructed by applying \emph{procedure} to each element
of the \emph{lists}.  The order of application is not defined.
\RRRRRS~27.

(\texttt{max} \emph{number} \emph{number} ...)\ returns the maximum of
its arguments.  \RRRRRS~21.

(\texttt{member} \emph{expression} \emph{list}) returns the first
\emph{sublist} of \emph{list} such that
(\texttt{equal?}\ \emph{expression} (\texttt{car} \emph{sublist})) is
true. If no match occurs, then \texttt{\#f} is returned.  \RRRRRS~17.

(\texttt{memq} \emph{expression} \emph{list}) returns the first
\emph{sublist} of \emph{list} such that
(\texttt{eq?}\ \emph{expression} (\texttt{car} \emph{sublist})) is
true. If no match occurs, then \texttt{\#f} is returned.  \RRRRRS~17.

(\texttt{memv} \emph{expression} \emph{list}) returns the first
\emph{sublist} of \emph{list} such that
(\texttt{eqv?}\ \emph{expression} (\texttt{car} \emph{sublist})) is
true. If no match occurs, then \texttt{\#f} is returned.  \RRRRRS~17.

(\texttt{min} \emph{number} \emph{number} ...)\ returns the minimum of
its arguments.  \RRRRRS~21.

(\texttt{module} \emph{symbol} \emph{clause} ...)\ \emph{syntax} to
declare module information for the \StoC\ compiler.  The \emph{module}
form must be the first item in the source file.  The module name is a
\emph{symbol} that must be a legal C identifier.  Using this
information, the compiler is able to construct an object module that
is similar in structure to a Modula 2 module. Following the module
name come optional \emph{clauses}. If the module is to provide the
``main'' program, then a \emph{clause} of the form (\texttt{main}
\emph{symbol}) is provided that indicates that \emph{symbol} is the
initial \emph{procedure}.  It will be invoked with one argument that
is a \emph{list} of \emph{strings} that are the arguments that the
program was invoked with.  A minimum (and default) heap size can be
specified by the \emph{clause} (\texttt{HEAP} \emph{integer}), where
the size is specified in megabytes.  The user may control that
top-level \emph{symbols} in this module are visible as top-level
\emph{symbols} by including a \emph{clause} of the form
(\texttt{top-level} \emph{symbol} ...). If this clause occurs, then
only those \emph{symbols} specified will be made top-level. All other
top-level \emph{symbols} in the module will appear at the top-level
with names of the form: \emph{module}\_\emph{symbol}.  If a
\texttt{top-level} clause is not provided, then all top-level
\emph{symbols} in the module will be made top-level. The final clause,
(\texttt{with} \emph{symbol} ...)\ indicates that this module will be
linked with other modules.  Normally the intermodule linkages are
automatically infered by including all \emph{modules} that have
external references. However, this mechanism is not sufficient to pick
up those objects that are only referenced at runtime.

(\texttt{modulo} \emph{integer}$_1$ \emph{integer}$_2$) returns the
modulo of its arguments.  The sign of the result is the sign of the
divisor.  \RRRRRS~22.

(\texttt{negative?}\ \emph{number}) \emph{predicate} that returns
\texttt{\#t} when \emph{number} is negative.  \RRRRRS~21.

(\texttt{newline} \emph{optional-output-port}) outputs a newline
character on the \emph{optional-output-port}.  \RRRRRS~31.

(\texttt{not} \emph{expression}) \emph{predicate} that returns
\texttt{\#t} when \emph{expression} is \texttt{\#f} or \texttt{()}.
\RRRRRS~13.

(\texttt{null?}\ \emph{expression}) \emph{predicate} that returns
\texttt{\#t} when \emph{expression} is \texttt{()}.  \RRRRRS~16.

\emph{number} \StoC\ has two internal representations for numbers:
\emph{fixed} and \emph{float}.  When an arithmetic operation is to be
performed with a \emph{float} argument, all arguments will be
converted to \emph{float} as needed, and then the operation will be
performed.  Automatic conversion back to \emph{fixed} is never done.
\RRRRRS~18.

(\texttt{number->string} \emph{number} \emph{format descriptor})
returns a \emph{string} that is the printed representation of
\emph{number} as specified by \emph{format descriptor}.  For
compatibility with \RRRRS.

(\texttt{number->string} \emph{number}) returns a string with the
printed representation of the number.  \RRRRRS~23.

(\texttt{number->string} \emph{number} \emph{radix}) returns a string
with the printed representation of the number in the given radix.
Radix must be 2, 8, 10, or 16.  \RRRRRS~23.

(\texttt{number?}\ \emph{expression}) \emph{predicate} that returns
\texttt{\#t} when \emph{expression} is a \emph{number}.  \RRRRRS~20.

(\texttt{odd?}\ \emph{integer}) \emph{predicate} that returns
\texttt{\#t} when \emph{integer} is odd.  \RRRRRS~21.

(\texttt{open-file} \emph{string}$_1$ \emph{string}$_2$) returns a
\emph{port} for file \emph{string}$_1$ that is opened using the
operating system's \emph{fopen} option \emph{string}$_2$.

(\texttt{open-input-file} \emph{string}) returns an \emph{input port}
capable of delivering characters from the file \emph{string}.
\RRRRRS~30.

(\texttt{open-input-string} \emph{string}) returns an \emph{input
  port} capable of delivering characters from the \emph{string}.

(\texttt{open-output-file} \emph{string}) returns an \emph{output
  port} capable of delivering characters to the file \emph{string}.
\RRRRRS~30.

(\texttt{open-output-string}) returns an \emph{output port} capable of
delivering characters to a \emph{string}.  See
\texttt{get-output-string}.

(\texttt{optimize-eval} \emph{option...})\ controls the optimization
done on interpreted programs.  When no \emph{option} is supplied,
minimal optimization is done.  When \texttt{call} is specified, calls
to top-level procedures that are not interpreted are optimized.  When
\texttt{rewrite} is specified, calls to top-level procedures that take
variable number of arguments are rewritten.  This option may cause
some breakpoints to be missed.  Both \texttt{call} and
\texttt{rewrite} may be specified.

\emph{optional-input-port} if present, it must be an
\emph{input-port}.  If not present, then it is the value returned by
\texttt{current-input-port}.

\emph{optional-output-port} if present, it must be an
\emph{output-port}.  If not present, then it is the value returned by
\texttt{current-output-port}.

(\texttt{or} \emph{expression} ...)\ \emph{syntax} for a conditional
expression.  \RRRRRS~9.

\emph{pair} record structure with two fields: car and cdr.
\RRRRRS~15.

(\texttt{pair?}\ \emph{expression}) \emph{predicate} that returns
\texttt{\#t} when \emph{expression} is a \emph{pair}.  \RRRRRS~16.

(\texttt{peek-char} \emph{optional-input-port}) returns a copy of the
next character available on \emph{optional-input-port}.  \RRRRRS~30.

\texttt{pointer} \emph{syntax} for declaring a non-Scheme procedure,
procedure argument, or global varible as being some type of C pointer.
When a value must be supplied, an expression of the type
\emph{string}, \emph{procedure}, or \emph{number} is supplied.  This
will result in either the address of the first character of the
\emph{string}, the address of the code associated with the
\emph{procedure}, or the value of the number being used.  A
\emph{pointer} value is returned as an non-negative \emph{number}.

\emph{port} Scheme object that is capable of delivering or accepting
characters on demand.  \RRRRRS~29.

(\texttt{port->stdio-file} \emph{port}) returns the standard I/O FILE
pointer for \emph{port}, or \texttt{\#f} if the \emph{port} does not
use standard I/O.

(\texttt{positive?}\ \emph{number}) \emph{predicate} that returns
\texttt{\#t} when \emph{number} is positive.  \RRRRRS~21.

(\texttt{pp} \emph{expression} \emph{optional-output-port})
pretty-prints \emph{expression} on \emph{optional-output-port}.

(\texttt{pp} \emph{expression} \emph{string}) pretty-prints
\emph{expression} to the file \emph{string}.

\emph{predicate} function that returns \texttt{\#t} when the condition
is true, and \texttt{\#f} when the condition is false.  \RRRRRS~13.

(\texttt{procedure?}\ \emph{expression}) \emph{predicate} that returns
\texttt{\#t} when \emph{expression} is a \emph{procedure}.
\RRRRRS~27.

(\texttt{proceed}) return from the innermost \texttt{read-eval-print}
loop with an unspecified value.

(\texttt{proceed}) resume the computation that previously timed out in
an embedded \StoC\ system, or was stopped at a breakpoint.

(\texttt{proceed} \emph{expression}) return from the innermost
\texttt{read-eval-print} loop with \emph{expression} as the value.  At
the outermost level, \emph{expression} must be an \emph{integer} as it
will be used as the argument for a call to the C library procedure
\emph{exit}.

(\texttt{proceed} \emph{expression}) return \emph{expression} as the
value of a procedure that stopped at a breakpoint.

(\texttt{proceed?}) force a breakpoint while resuming the computation
that previously timed out in an embedded \StoC\ system.

(\texttt{putprop} \emph{symbol} \emph{expression}$_1$
\emph{expression}$_2$) stores \emph{expression}$_2$ using key
\emph{expression}$_1$ on \emph{symbol's} property list.  See
\texttt{getprop}.

(\texttt{quasiquote} \emph{back-quote-template}) \emph{syntax} for a
\emph{vector} or \emph{list} constructor.  \RRRRRS~11.

(\texttt{quote} \emph{expression}) \emph{syntax} whose result is
\emph{expression}.  \RRRRRS~7.

(\texttt{quotient} \emph{integer}$_1$ \emph{integer}$_2$) returns the
quotient of its arguments.  The sign is the sign of the product of its
arguments. \RRRRRS~22.

(\texttt{rational?}\ \emph{number}) predicate that returns
\texttt{\#t} when its argument is a rational \emph{number}.  This is
true for any number in \StoC.  \RRRRRS~20.

(\texttt{read} \emph{optional-input-port}) returns the next readable
object from \emph{optional-input-port}.  Revived$^3$~30.

(\texttt{read-char} \emph{optional-input-port}) returns the next
character from \emph{optional-input-port}, updating the \emph{port} to
point to the next \emph{character}.  Revived$^3$~30.

(\texttt{read-eval-print} \emph{expression} ...)\ starts a new
read-eval-print loop.  The optional \emph{expressions} allow one to
specify the prompt or the header: \texttt{PROMPT} \emph{string}
\texttt{HEADER} \emph{string}.  Typing control-D at the prompt will
terminate the procedure.  See \texttt{reset}, \texttt{exit},
\texttt{eval}, \texttt{proceed}.

(\texttt{real?}\ \emph{number}) predicate that returns \texttt{\#t}
when its argument is a real \emph{number}.  This is true in \StoC\ for
any \emph{number}. \RRRRRS~20.

\emph{record} a heterogenous mutable structure whose elements are
indexed by \emph{integers}.  The valid indexes of a record are the
exact non-negative integers less than the length of the record. A
\emph{record} differs from a \emph{vector} in that a \emph{record} may
have method \emph{procedures} that control how it's output, compared,
and evaluated.

(\texttt{remainder} \emph{integer}$_1$ \emph{integer}$_2$) returns the
remainder of its arguments.  The sign is the sign of
\emph{integer}$_1$. \RRRRRS~22.

(\texttt{remove} \emph{expression} \emph{list}) returns a new
\emph{list} that is a copy of \emph{list} with all items
\texttt{equal?}\ to \emph{expression} removed from it.

(\texttt{remove!}\ \emph{expression} \emph{list}) returns \emph{list}
having deleted all items \texttt{equal?}\ to \emph{expression} from
it.

(\texttt{remove-file} \emph{string}) removes the file named
\emph{string}.

(\texttt{remq} \emph{expression} \emph{list}) returns a new
\emph{list} that is a copy of \emph{list} with all items
\texttt{eq?}\ to \emph{expression} removed from it.

(\texttt{remq!}\ \emph{expression} \emph{list}) returns \emph{list}
having deleted all items \texttt{eq?}\ to \emph{expression} from it.

(\texttt{remv} \emph{expression} \emph{list}) returns a new
\emph{list} that is a copy of \emph{list} with all items
\texttt{eqv?}\ to \emph{expression} removed from it.

(\texttt{remv!}\ \emph{expression} \emph{list}) returns \emph{list}
having deleted all items \texttt{eqv?}\ to \emph{expression} from it.

(\texttt{rename-file} \emph{string}$_1$ \emph{string}$_2$) changes the
name of the file named \emph{string}$_1$ to \emph{string}$_2$.

(\texttt{reset}) returns to the current \texttt{read-eval-print} loop.

(\texttt{reset-bpt}) indicates that the caller wishes to cancel the
resumption of computation at the point where a breakpoint occurred in
an embedded \StoC\ system.

(\texttt{reset-error}) indicates that the caller is finished examining
the last retained error state in an embedded \StoC\ system.

(\texttt{reverse} \emph{list}) returns a new \emph{list} with the
elements of \emph{list} in reverse order.  \RRRRRS~17.

(\texttt{round} \emph{number}) returns \emph{number} rounded to the
closest integer.  \RRRRRS~22.

\texttt{S2CUINT} C type defined by \StoC\ to be an unsigned integer
that is the same size as a pointer.

\emph{sc-pointer} a Scheme object that is represented by a tagged
pointer to one or more words of memory.

\texttt{sc...}\ all modules that compose the \StoC\ runtime system
have module names begining with the letters \texttt{sc}.  All
procedures and external variables in these modules have names that
begin with \texttt{sc...\_}.

\texttt{s2cc} shell command to invoke the \StoC\ Scheme compiler.  See
the \texttt{man} page.

\texttt{S2CGCINFO} environment variable that when set to 1 will log
garbage collection information on stderr.  This variable is overridden
by the \texttt{-scgc} command line flag.

\texttt{S2CHEAP} environment variable that controls the initial heap
size.  It is set to the desired size in megabytes.  If not set, then
the default in the main program will be used.  If a default size is
not supplied, then the implementation default is used.  This variable
is overridden by the \texttt{-sch} command line flag.

\texttt{S2CLIMIT} environment variable that controls the amount of heap
retained after a generational garbage collection that will force a
full collection.  It is expressed as a percent of the heap.  The
default value is 40.  This variable is overridden by the \texttt{-scl}
command line flag.

\texttt{S2CMAXHEAP} environment variable that controls the maximum heap
size.  It is set to the desired size in megabytes.  If not set and the
\texttt{-scmh} command line flag is not supplied, the maximum heap
size is five times the initial heap size. This variable is overridden
by the \texttt{-scmh} command line flag.

(\texttt{scheme-byte-ref} \emph{sc-pointer} \emph{integer}) returns
the byte at the \emph{integer} byte of \emph{sc-pointer} as a
\emph{number}.

(\texttt{scheme-byte-set!}\ \emph{sc-pointer} \emph{integer}
\emph{number}) sets the byte at the \emph{integer} byte of
\emph{sc-pointer} to \emph{number}.  The procedure returns
\emph{number} as its value.

(\texttt{scheme-int-ref} \emph{sc-pointer} \emph{integer}) return the
int at the \emph{integer} byte of \emph{sc-pointer} as a
\emph{number}.

(\texttt{scheme-int-set!}\ \emph{sc-pointer} \emph{integer}
\emph{number}) sets the int at the \emph{integer} byte of
\emph{sc-pointer} to \emph{number}.  The procedure returns
\emph{number} as its value.

(\texttt{scheme-s2cuint-ref} \emph{sc-pointer} \emph{integer}) returns
the S2CUINT at the \emph{integer} byte of \emph{sc-pointer}.

(\texttt{scheme-s2cuint-set!}\ \emph{sc-pointer} \emph{integer}
\emph{expression}) sets the S2CUINT at the \emph{integer} byte of
\emph{sc-pointer} to \emph{expression}.  The procedure returns
\emph{expression} as its value.

(\texttt{scheme-tscp-ref} \emph{sc-pointer} \emph{integer}) returns
the TSCP at the \emph{integer} byte of \emph{sc-pointer}.

(\texttt{scheme-tscp-set!}\ \emph{sc-pointer} \emph{integer}
\emph{expression}) sets the TSCP at the \emph{integer} byte of
\emph{sc-pointer} to \emph{expression}.  The procedure returns
\emph{expression} as its value.

\texttt{s2ci} shell command to invoke the \StoC\ Scheme interpreter.
See the \texttt{man} page.

\texttt{s2ci} \emph{format descriptor} for compatibility with \RRRRS.

(\texttt{set!}\ \emph{symbol} \emph{expression}) \emph{syntax} to set
the location bound to \emph{symbol} to the value of \emph{expression}.
\RRRRRS~9.

(\texttt{set-car!}\ \emph{pair} \emph{expression}) sets the car field
of \emph{pair} to \emph{expression}.  \RRRRRS~16.

(\texttt{set-cdr!}\ \emph{pair} \emph{expression}) sets the cdr field
of \emph{pair} to \emph{expression}.  \RRRRRS~16.

(\texttt{set-gcinfo!}\ \emph{integer}) sets the flag controlling the
printing of garbage collection statistics to \emph{integer}.  See
\texttt{-scgc}.

(\texttt{set-generation-limit!}\ \emph{integer}) sets the full
collection limit to \emph{integer}.  See \texttt{-scl}.

(\texttt{set-maximum-heap!}\ \emph{integer}) sets the maximum heap
size to \emph{integer} megabytes.  See \texttt{-scmh}.

(\texttt{set-stack-size!}\ \emph{expression}) sets the size of the
stack used by \StoC\ to \emph{expression} bytes.  This value is
ignored if the system does not do explicit stack overflow checking.

(\texttt{set-time-slice!}\ \emph{expression}) sets the time slice used
by the \StoC\ to \emph{expression} ticks. This value is decremented
each time a Scheme procedure is called, and the time slice expires
when it becomes zero.  This value is ignored if the system does not do
explicit time slicing.

(\texttt{set-top-level-value!}\ \emph{symbol} \emph{expression}) sets
the top-level location bound to \emph{symbol} to value.

(\texttt{set-write-circle!}\ \emph{boolean}
\emph{optional-output-port}) controls circular object detection on
output to \emph{optional-output-port}. If \emph{boolean} is
\texttt{\#t}, then circular objects are printed as ``...''.  If
\emph{boolean} is \texttt{\#f}, circular object detection is disabled.

(\texttt{set-write-length!}\ \emph{integer}
\emph{optional-output-port}) sets the list and vector length limits of
\emph{optional-output-port} to \emph{integer}.  Vectors and lists
longer than \emph{integer} have their remaining elements printed as
``...''.

(\texttt{set-write-length!}\ \texttt{\#f} \emph{optional-output-port})
allows arbitrary length list and vector printing on
\emph{optional-output-port}.

(\texttt{set-write-level!}\ \emph{integer}
\emph{optional-output-port}) sets the number of levels that nested
vectors and lists are printed on \emph{optional-output-port} to
\emph{integer}.  Vectors and lists nesting deeper than this level are
printed as ``\#''.

(\texttt{set-write-level!}\ \texttt{\#f} \emph{optional-output-port})
allows arbitrarily deep nested list and vector printing on
\emph{optional-output-port}.

(\texttt{set-write-pretty!}\ \emph{boolean}
\emph{optional-output-port}) controls ``pretty-printing'' on
\emph{optional-output-port}.  If \emph{boolean} is \texttt{\#t}, then
output is printed in a more readable form in \texttt{write-width} wide
lines.  A value of \texttt{\#f} enables normal output.

(\texttt{set-write-width!}\ \emph{integer}
\emph{optional-output-port}) sets the width of
\emph{optional-output-port} to \emph{integer}.

\texttt{shortint} \emph{syntax} for declaring a non-Scheme procedure,
procedure argument, or global variable as the C type \texttt{short
  int}.  When a \texttt{short int} value must be supplied, an
expression of type \emph{number} must be supplied.  When a
\texttt{short int} value is returned, a value of type \emph{number} is
returned.

\texttt{shortunsigned} \emph{syntax} for declaring a non-Scheme
procedure, procedure argument, or global variable as the C type
\texttt{short unsigned}. When a \texttt{unsigned short} value must be
supplied, an expression of type \emph{number} must be supplied.  When
a \texttt{short unsigned} value is returned, a value of type
\emph{number} is returned.

(\texttt{sin} \emph{number}) returns the sine of its argument.
\RRRRRS~23.

(\texttt{signal} \emph{number} \emph{expression}) provides a signal
handler for the operating system dependent signal \emph{number}.  The
\emph{expression} is the signal handler and is either a
\emph{procedure} or a \emph{number}.  When a procedure is supplied, it
is called with the signal number when the signal is present.  Numeric
handler values are interpreted by the underlying operating system.
The previous value of the signal handler is returned.

(\texttt{sqrt} \emph{number}) returns the square root of its argument.
\RRRRRS~23.

(\texttt{stack-size}) returns the size in bytes of \StoC's stack.

\texttt{stderr-port} \emph{port} to output characters to stderr.

\texttt{stdin-port} \emph{port} to input characters from stdin.

\texttt{stdout-port} \emph{port} to output characters to stdout.

\emph{string} sequence of \emph{characters}.  The valid indexes of a
\emph{string} are exact non-negative integers less than the length of
the string.\RRRRRS~25.

(\texttt{string} \emph{char} ...)\ returns a newly allocated
\emph{string} whose elements contain the given arguments.  \RRRRRS~25.

(\texttt{string->list} \emph{string}) returns a newly constructed
\emph{list} that contains the elements of \emph{string}.  \RRRRRS~25.

(\texttt{string->number} \emph{string}) returns a number expressed by
\emph{string}.  If \emph{string} is not a syntactically valid notation
for a number then it returns \texttt{\#f}.  \RRRRRS~24.

(\texttt{string->number} \emph{string} \emph{number}) returns a number
expressed by \emph{string} with \emph{number} the default radix.
Radix must be 2, 8, 10, or 16. If \emph{string} is not a syntactically
valid notation for a number then it returns \texttt{\#f}.  \RRRRRS~24.

(\texttt{string->symbol} \emph{string}) returns the interned
\emph{symbol} whose name is \emph{string}.  \RRRRRS~18.

(\texttt{string->uninterned-symbol} \emph{string}) returns an
uninterned \emph{symbol} whose name is string.

(\texttt{string-append} \emph{string} \emph{string} ...)\ returns a
new \emph{string} whose \emph{characters} are the concatenation of the
of the given \emph{strings}. Upper and lower case letters are treated
as though they were the same character. \RRRRRS~26.

(\texttt{string-ci<=?}\ \emph{string}$_1$ \emph{string}$_2$)
\emph{predicate} that returns \texttt{\#t} when \emph{string}$_1$ is
less than or equal to \emph{string}$_2$.  Upper and lower case letters
are treated as though they were the same character. \RRRRRS~26.

(\texttt{string-ci<?}\ \emph{string}$_1$ \emph{string}$_2$)
\emph{predicate} that returns \texttt{\#t} when \emph{string}$_1$ is
less than \emph{string}$_2$.  Upper and lower case letters are treated
as though they were the same character. \RRRRRS~26.

(\texttt{string-ci=?}\ \emph{string}$_1$ \emph{string}$_2$)
\emph{predicate} that returns \texttt{\#t} when \emph{string}$_1$ is
equal to \emph{string}$_2$.  Upper and lower case letters are treated
as though they were the same character. \RRRRRS~26.

(\texttt{string-ci>=?}\ \emph{string}$_1$ \emph{string}$_2$)
\emph{predicate} that returns \texttt{\#t} when \emph{string}$_1$ is
greater than or equal to \emph{string}$_2$.  Upper and lower case
letters are treated as though they were the same
character. \RRRRRS~26.

(\texttt{string-ci>?}\ \emph{string}$_1$ \emph{string}$_2$)
\emph{predicate} that returns \texttt{\#t} when \emph{string}$_1$ is
greater than \emph{string}$_2$.  Upper and lower case letters are
treated as though they were the same character. \RRRRRS~26.

(\texttt{string-copy} \emph{string}) returns a new \emph{string} whose
\emph{characters} are those of the given \emph{string}.  \RRRRRS~26.

(\texttt{string-fill!}\ \emph{string} \emph{char}) stores \emph{char}
in every element of \emph{string}.  \RRRRRS~26.

(\texttt{string-length} \emph{string}) returns the length of
\emph{string}.  \RRRRRS~25.

(\texttt{string-ref} \emph{string} \emph{integer}) returns
\emph{character} that is the \emph{integer} element of \emph{string}.
The first element is 0. \RRRRRS~25.

(\texttt{string-set!}\ \emph{string} \emph{integer} \emph{character})
sets the \emph{integer} element of \emph{string} to \emph{character}.
The first element is 0.  \RRRRRS~26.

(\texttt{string<=?}\ \emph{string}$_1$ \emph{string}$_2$)
\emph{predicate} that returns \texttt{\#t} when \emph{string}$_1$ is
less than or equal to \emph{string}$_2$.  \RRRRRS~26.

(\texttt{string<?}\ \emph{string}$_1$ \emph{string}$_2$)
\emph{predicate} that returns \texttt{\#t} when \emph{string}$_1$ is
less than \emph{string}$_2$.  \RRRRRS~26.

(\texttt{string=?}\ \emph{string}$_1$ \emph{string}$_2$)
\emph{predicate} that returns \texttt{\#t} when \emph{string}$_1$ is
equal to \emph{string}$_2$.  \RRRRRS~26.

(\texttt{string>=?}\ \emph{string}$_1$ \emph{string}$_2$)
\emph{predicate} that returns \texttt{\#t} when \emph{string}$_1$ is
greater than or equal to \emph{string}$_2$.  \RRRRRS~26.

(\texttt{string>?}\ \emph{string}$_1$ \emph{string}$_2$)
\emph{predicate} that returns \texttt{\#t} when \emph{string}$_1$ is
greater than \emph{string}$_2$.  \RRRRRS~26.

(\texttt{string?}\ \emph{expression}) \emph{predicate} that returns
\texttt{\#t} when \emph{expression} is a \emph{string}.  \RRRRRS~25.

(\texttt{substring} \emph{string} \emph{integer}$_1$
\emph{integer}$_2$) returns a \emph{string} consisting of
\emph{integer}$_2$-\emph{integer}$_1$ elements of \emph{string}
starting at element \emph{integer}$_1$.  \RRRRRS~26.

(\texttt{symbol?}\ \emph{expression}) \emph{predicate} that returns
\texttt{\#t} when \emph{expression} is a \emph{symbol}.  \RRRRRS~18.

(\texttt{symbol->string} \emph{symbol}) returns the name of
\emph{symbol} as a \emph{string}.  \RRRRRS~18.

\emph{syntax} indicates a form that is evaluated in a manner that is
specific to the form.  \RRRRRS~5.

(\texttt{system} \emph{string}) issue the shell command contained in
\emph{string} and return the result.  See the man page for the
\texttt{system} procedure for details.

(\texttt{tan} \emph{number}) returns the tangent of its argument.
\RRRRRS~23.

(\texttt{time-of-day}) returns a system dependent \emph{string}
representing the current time and date.

(\texttt{time-slice}) returns the current time slice value.

(\texttt{top-level}) returns control to the ``top-level''
\texttt{read-eval-print} loop.

(\texttt{top-level-value} \emph{symbol}) returns the value in the
location that is the ``top-level'' binding of \emph{symbol}.

(\texttt{trace}) returns a list of the procedures being traced.

(\texttt{trace} \emph{symbol} \emph{symbol} ...)\ enables tracing on
the \emph{procedures} that are the values of the \emph{symbols}.

\texttt{trace-output-port} \emph{port} used for trace output. The
default value is the same as \texttt{stdout-port}.

(\texttt{transcript-off}) turns off the transcript.  \RRRRRS~31.

(\texttt{transcript-on} \emph{string}) starts a transcript on the file
\emph{string}.  \RRRRRS~31.

(\texttt{truncate} \emph{number}) returns the truncated value of
\emph{number}.  \RRRRRS~22.

\texttt{tscp} \emph{syntax} for declaring a non-Scheme procedure,
procedure argument, or global variable as the C type
\texttt{TSCP}. The type \texttt{TSCP} is a tagged pointer to a Scheme
object. When a \texttt{tscp} value must be supplied, any expression
may be supplied.  When a \texttt{tscp} value is returned, any type of
value may be returned.

(\texttt{unbpt}) \emph{syntax} to remove all breakpoints.

(\texttt{unbpt} \emph{symbol} \emph{symbol} ...)\ \emph{syntax} to
remove breakpoints from the named \emph{procedures}.

(\texttt{uninterned-symbol?}\ \emph{symbol}) \emph{predicate} that
returns \texttt{\#t} if \emph{symbol} is not \emph{interned}.

(\texttt{unless} \emph{expression}$_1$ \emph{expression}$_2$
...)\ \emph{syntax} for a conditional form that is equivalent to
(\texttt{if} (\texttt{not} \emph{expression}$_1$) (\texttt{begin}
\emph{expression}$_2$ ...)).

(\texttt{unquote} \emph{expression}) \emph{syntax} to evaluate the
expression and replaces it in the \emph{back-quote-template}.
\RRRRRS~12.

(\texttt{unquote-splicing} \emph{expression}) \emph{syntax} to
evaluate the expression and splices it into the
\emph{back-quote-template}.  \RRRRRS~12.

\texttt{unsigned} \emph{syntax} for declaring a non-Scheme procedure,
procedure argument, or global variable as the C type
\texttt{unsigned}.  When a \texttt{unsigned} value must be supplied,
an expression of type \emph{number} must be supplied.  When a
\texttt{unsigned} value is returned, a value of type \emph{number} is
returned.

(\texttt{untrace}) \emph{syntax} to remove tracing from all
\emph{procedures}.

(\texttt{untrace} \emph{symbol} \emph{symbol} ...)\ \emph{syntax} to
remove tracing from the named \emph{procedures}.

\emph{variable} \RRRRRS~6.

\emph{vector} a heterogenous mutable structure whose elements are
indexed by \emph{integers}.  The valid indexes of a vector are the
exact non-negative integers less than the length of the
vector. \RRRRRS~26.

(\texttt{vector} \emph{expression} ...)\ returns a newly allocated
\emph{vector} whose elements contain the given arguments.  \RRRRRS~27.

(\texttt{vector-fill!}\ \emph{vector} \emph{expression}) stores
\emph{expression} in every element of \emph{vector}.  \RRRRRS~27.

(\texttt{vector-length} \emph{vector}) returns the number of elements
in \emph{vector}.  \RRRRRS~27.

(\texttt{vector->list} \emph{vector}) returns a newly created
\emph{list} of the objects contained in the elements of the
\emph{vector}.  \RRRRRS~27.

(\texttt{vector-ref} \emph{vector} \emph{integer}) returns the
contents of element \emph{integer} of \emph{vector}.  The first
element is 0.  \RRRRRS~27.

(\texttt{vector-set!}\ \emph{vector} \emph{integer} \emph{expression})
sets element \emph{integer} of \emph{vector} to \emph{expression}.
The first element is 0.  \RRRRRS~27.

(\texttt{vector?}\ \emph{expression}) \emph{predicate} that returns
\texttt{\#t} when \emph{expression} is a \emph{vector}.  \RRRRRS~26.

\texttt{void} \emph{syntax} for declaring a non-Scheme procedure as
returning the C type \texttt{void}.  The value of such a procedure may
not be used.

(\texttt{wait-system-file} \emph{expression}) waits for input on the
file with the system file number \emph{expression}.  When input is
available, the procedure returns.  If \emph{expression} is equal to
\texttt{\#f}, then the procedure will not return until all tasks have
been completed.

(\texttt{weak-cons} \emph{expression}$_1$ \emph{expression}$_2$)
returns a newly allocated \emph{pair} that has \emph{expression}$_1$
as its \texttt{car}, and \emph{expression}$_2$ as its \texttt{cdr}.
If the garbage collector discovers that pointers to an object only
exist in the \texttt{car}'s of \emph{pair}s created by
\texttt{weak-cons}, then it may recover the object and set the
\texttt{car}'s in those \emph{pair}s to \texttt{\#f}.

(\texttt{when} \emph{expression}$_1$ \emph{expression}$_2$
...)\ \emph{syntax} for a conditional form that is equivalent to
(\texttt{if} \emph{expression}$_1$ (\texttt{begin}
\emph{expression}$_2$ ...)).

(\texttt{when-unreferenced} \emph{expression} \emph{procedure})
applies the clean-up procedure \emph{procedure} (with the object
represented by \emph{expression} as its argument) at some point in the
future when the object represented by \emph{expression} is no longer
referenced by the program.  The procedure returns either the cleanup
procedure supplied by an earlier call to \texttt{when-unreferenced},
or \texttt{\#f} when no cleanup procedure was defined.

(\texttt{when-unreferenced} \emph{expression} \texttt{\#f}) returns
either the cleanup procedure for the object represented by
\emph{expression} or \texttt{\#f} when no cleanup procedure was
defined.  In either case, the Scheme system will take no action when
the object represented by \emph{expression} is no longer referenced by
the program.

(\texttt{with-input-from-file} \emph{string} \emph{procedure}) opens
the file \emph{string}, makes its \emph{port} the default
\emph{input-port}, then calls \emph{procedure} with no arguments.
\RRRRRS~30.

(\texttt{with-output-to-file} \emph{string} \emph{procedure}) opens
the file \emph{string}, makes its \emph{port} the default
\emph{output-port}, then calls \emph{procedure} with no arguments.
\RRRRRS~30.

(\texttt{write} \emph{expression} \emph{optional-output-port}) outputs
\emph{expression} to \emph{optional-output-port} in a machine-readable
form.  \RRRRRS~31.

(\texttt{write-char} \emph{character} \emph{optional-output-port})
outputs \emph{character} to \emph{optional-output-port}.  \RRRRRS~31.

(\texttt{write-circle} \emph{optional-output-port}) returns a
\emph{boolean} indicating whether circular objects are detected when
output to \emph{optional-output-port}.

(\texttt{write-count} \emph{optional-output-port}) returns the number
of characters on the current line in \emph{optional-output-port}.

(\texttt{write-length} \emph{optional-output-port}) returns either an
\emph{integer} indicating the maximum length vector or list printed on
\emph{optional-output-port}, or \texttt{\#f} indicating that arbitrary
length objects are printed on \emph{optional-output-port}.

(\texttt{write-level} \emph{optional-output-port}) returns either an
\emph{integer} indicating the maximum nesting depth of objects that
are printed on \emph{optional-output-port}, or \texttt{\#f} indicating
that arbitrary depth objects are printed on
\emph{optional-output-port}.

(\texttt{write-pretty} \emph{optional-output-port}) returns a
\emph{boolean} indicating whether pretty-printing is done on
\emph{optional-output-port}.

(\texttt{write-width} \emph{optional-output-port}) returns the width
of \emph{optional-output-port} in \emph{characters}.

(\texttt{zero?}\ \emph{number}) predicate that returns \texttt{\#t}
when \emph{number} is zero.  \RRRRRS~21.

\texttt{\texttildelow\%} \emph{format descriptor} to output a newline
character.

\texttt{\texttildelow\texttildelow} \emph{format descriptor} to output
a \texttt{\texttildelow}.

\texttt{\texttildelow{}A} \emph{format descriptor} to output the next
\emph{expression} using \texttt{display}.

\texttt{\texttildelow{}a} \emph{format descriptor} identical to
\texttt{\texttildelow{}A}.

\texttt{\texttildelow{}C} \emph{format descriptor} to output the next
\emph{expression} (that must be a \emph{character}) using
\texttt{write-char}.

\texttt{\texttildelow{}c} \emph{format descriptor} identical to
\texttt{\texttildelow{}C}.

\texttt{\texttildelow{}S} \emph{format descriptor} to output the next
\emph{expression} using \texttt{write}.

\texttt{\texttildelow{}s} \emph{format descriptor} identical to
\texttt{\texttildelow{}S}.
\end{document}
